\documentclass[]{article}
\usepackage{lmodern}
\usepackage{amssymb,amsmath}
\usepackage{ifxetex,ifluatex}
\usepackage{fixltx2e} % provides \textsubscript
\ifnum 0\ifxetex 1\fi\ifluatex 1\fi=0 % if pdftex
  \usepackage[T1]{fontenc}
  \usepackage[utf8]{inputenc}
\else % if luatex or xelatex
  \ifxetex
    \usepackage{mathspec}
  \else
    \usepackage{fontspec}
  \fi
  \defaultfontfeatures{Ligatures=TeX,Scale=MatchLowercase}
\fi
% use upquote if available, for straight quotes in verbatim environments
\IfFileExists{upquote.sty}{\usepackage{upquote}}{}
% use microtype if available
\IfFileExists{microtype.sty}{%
\usepackage{microtype}
\UseMicrotypeSet[protrusion]{basicmath} % disable protrusion for tt fonts
}{}
\usepackage[margin=1in]{geometry}
\usepackage{hyperref}
\hypersetup{unicode=true,
            pdftitle={Propiedades de la población},
            pdfauthor={Carlos Iván Espinosa},
            pdfborder={0 0 0},
            breaklinks=true}
\urlstyle{same}  % don't use monospace font for urls
\usepackage{color}
\usepackage{fancyvrb}
\newcommand{\VerbBar}{|}
\newcommand{\VERB}{\Verb[commandchars=\\\{\}]}
\DefineVerbatimEnvironment{Highlighting}{Verbatim}{commandchars=\\\{\}}
% Add ',fontsize=\small' for more characters per line
\usepackage{framed}
\definecolor{shadecolor}{RGB}{248,248,248}
\newenvironment{Shaded}{\begin{snugshade}}{\end{snugshade}}
\newcommand{\KeywordTok}[1]{\textcolor[rgb]{0.13,0.29,0.53}{\textbf{{#1}}}}
\newcommand{\DataTypeTok}[1]{\textcolor[rgb]{0.13,0.29,0.53}{{#1}}}
\newcommand{\DecValTok}[1]{\textcolor[rgb]{0.00,0.00,0.81}{{#1}}}
\newcommand{\BaseNTok}[1]{\textcolor[rgb]{0.00,0.00,0.81}{{#1}}}
\newcommand{\FloatTok}[1]{\textcolor[rgb]{0.00,0.00,0.81}{{#1}}}
\newcommand{\ConstantTok}[1]{\textcolor[rgb]{0.00,0.00,0.00}{{#1}}}
\newcommand{\CharTok}[1]{\textcolor[rgb]{0.31,0.60,0.02}{{#1}}}
\newcommand{\SpecialCharTok}[1]{\textcolor[rgb]{0.00,0.00,0.00}{{#1}}}
\newcommand{\StringTok}[1]{\textcolor[rgb]{0.31,0.60,0.02}{{#1}}}
\newcommand{\VerbatimStringTok}[1]{\textcolor[rgb]{0.31,0.60,0.02}{{#1}}}
\newcommand{\SpecialStringTok}[1]{\textcolor[rgb]{0.31,0.60,0.02}{{#1}}}
\newcommand{\ImportTok}[1]{{#1}}
\newcommand{\CommentTok}[1]{\textcolor[rgb]{0.56,0.35,0.01}{\textit{{#1}}}}
\newcommand{\DocumentationTok}[1]{\textcolor[rgb]{0.56,0.35,0.01}{\textbf{\textit{{#1}}}}}
\newcommand{\AnnotationTok}[1]{\textcolor[rgb]{0.56,0.35,0.01}{\textbf{\textit{{#1}}}}}
\newcommand{\CommentVarTok}[1]{\textcolor[rgb]{0.56,0.35,0.01}{\textbf{\textit{{#1}}}}}
\newcommand{\OtherTok}[1]{\textcolor[rgb]{0.56,0.35,0.01}{{#1}}}
\newcommand{\FunctionTok}[1]{\textcolor[rgb]{0.00,0.00,0.00}{{#1}}}
\newcommand{\VariableTok}[1]{\textcolor[rgb]{0.00,0.00,0.00}{{#1}}}
\newcommand{\ControlFlowTok}[1]{\textcolor[rgb]{0.13,0.29,0.53}{\textbf{{#1}}}}
\newcommand{\OperatorTok}[1]{\textcolor[rgb]{0.81,0.36,0.00}{\textbf{{#1}}}}
\newcommand{\BuiltInTok}[1]{{#1}}
\newcommand{\ExtensionTok}[1]{{#1}}
\newcommand{\PreprocessorTok}[1]{\textcolor[rgb]{0.56,0.35,0.01}{\textit{{#1}}}}
\newcommand{\AttributeTok}[1]{\textcolor[rgb]{0.77,0.63,0.00}{{#1}}}
\newcommand{\RegionMarkerTok}[1]{{#1}}
\newcommand{\InformationTok}[1]{\textcolor[rgb]{0.56,0.35,0.01}{\textbf{\textit{{#1}}}}}
\newcommand{\WarningTok}[1]{\textcolor[rgb]{0.56,0.35,0.01}{\textbf{\textit{{#1}}}}}
\newcommand{\AlertTok}[1]{\textcolor[rgb]{0.94,0.16,0.16}{{#1}}}
\newcommand{\ErrorTok}[1]{\textcolor[rgb]{0.64,0.00,0.00}{\textbf{{#1}}}}
\newcommand{\NormalTok}[1]{{#1}}
\usepackage{graphicx,grffile}
\makeatletter
\def\maxwidth{\ifdim\Gin@nat@width>\linewidth\linewidth\else\Gin@nat@width\fi}
\def\maxheight{\ifdim\Gin@nat@height>\textheight\textheight\else\Gin@nat@height\fi}
\makeatother
% Scale images if necessary, so that they will not overflow the page
% margins by default, and it is still possible to overwrite the defaults
% using explicit options in \includegraphics[width, height, ...]{}
\setkeys{Gin}{width=\maxwidth,height=\maxheight,keepaspectratio}
\IfFileExists{parskip.sty}{%
\usepackage{parskip}
}{% else
\setlength{\parindent}{0pt}
\setlength{\parskip}{6pt plus 2pt minus 1pt}
}
\setlength{\emergencystretch}{3em}  % prevent overfull lines
\providecommand{\tightlist}{%
  \setlength{\itemsep}{0pt}\setlength{\parskip}{0pt}}
\setcounter{secnumdepth}{0}
% Redefines (sub)paragraphs to behave more like sections
\ifx\paragraph\undefined\else
\let\oldparagraph\paragraph
\renewcommand{\paragraph}[1]{\oldparagraph{#1}\mbox{}}
\fi
\ifx\subparagraph\undefined\else
\let\oldsubparagraph\subparagraph
\renewcommand{\subparagraph}[1]{\oldsubparagraph{#1}\mbox{}}
\fi

%%% Use protect on footnotes to avoid problems with footnotes in titles
\let\rmarkdownfootnote\footnote%
\def\footnote{\protect\rmarkdownfootnote}

%%% Change title format to be more compact
\usepackage{titling}

% Create subtitle command for use in maketitle
\providecommand{\subtitle}[1]{
  \posttitle{
    \begin{center}\large#1\end{center}
    }
}

\setlength{\droptitle}{-2em}

  \title{Propiedades de la población}
    \pretitle{\vspace{\droptitle}\centering\huge}
  \posttitle{\par}
    \author{Carlos Iván Espinosa}
    \preauthor{\centering\large\emph}
  \postauthor{\par}
      \predate{\centering\large\emph}
  \postdate{\par}
    \date{4 de octubre de 2019}


\begin{document}
\maketitle

{
\setcounter{tocdepth}{2}
\tableofcontents
}
\begin{center}\rule{0.5\linewidth}{\linethickness}\end{center}

\section{Introducción}\label{introduccion}

Nosotros como individuos que tan consientes somos de pertenecer a un
grupo más grande de individuos, una población, y que de una u otra
manera las características de ese grupo tiene un efecto sobre lo que nos
pasa a nosotros individualmente. Pensemos en una propiedad del individuo
la cantidad de hijos, es igual la cantidad de hijos que ahora tienen las
parejas en relación a lo que tenían los abuelos de estas. Al igual que
la población humana, las poblaciones de otros organismos tienen efectos
sobre los organismos y esos efectos a su vez tienen un impacto sobre las
propiedades de la población. Durante los últimos años los cambios que se
están suscitando alrededor del planeta están afectando a las poblaciones
de diversas especies. Poder entender como una población responde a esas
presiones es fundamental si queremos emprender acciones de conservación.
Este análisis pasa por la posibilidad de describir eficientemente la
población; ¿Dónde se puede establecer la población? ¿Cuál es la
abundancia de esa población bajo determinadas condiciones? En el
presente ejercicio trabajaremos en comprender las herramientas
existentes para realizar la caracterización de la comunidad y sobre todo
en desarrollar habilidades para realizar una interpretación biológica de
esa caracterización. Una primera restricción a la cual se enfrentan los
organismos es el ambiente. Las condiciones ambientales restringe donde
una población puede establecerse. La capacidad de adaptación al ambiente
que tienen las diferentes especies define los rangos de distribución,
hay algunos animales que tienen rangos amplios de distribución y otros
que tienen unos rangos restringidos. De esta forma el ambiente determina
en primera medida la \textbf{\emph{distribución geográfica}} de una
especie, su \emph{nicho fundamental}. Una vez la población se ha
establecido, una propiedad importante es su tamaño poblacional, la
\textbf{\emph{abundancia poblacional}}. La abundancia de la población
está definida por dos factores, por un lado, el rango de distribución y
por el otro la \emph{densidad de la población}. La distribución en el
espacio de la población no es necesariamente homogéneo y este depende de
la escala a la cual estemos midiendo. De esta forma, las poblaciones
ocupan el espacio generando un \textbf{\emph{patrón espacial}}, que a su
vez determina la densidad y la abundancia.

\begin{center}\rule{0.5\linewidth}{\linethickness}\end{center}

\section{Distribución geográfica}\label{distribucion-geografica}

Como mencionamos anteriormente la distribución geográfica de una
población está definida en primera medida por las condiciones
ambientales, las especies ocurren ahí donde las condiciones
macro-climáticas le permiten estar. Sin embargo, las especies no
necesariamente se encuentran ocupando todo ese espacio, otros factores
pueden limitar su distribución. Las barreras geográficas, por ejemplo,
limitan las áreas que estas especies pueden realmente ocupar.
Adicionalmente, las especies pueden ser excluidas de sitios
ambientalmente adecuados por competencia con otras especies.

Durante los últimos años se han desarrollado múltiples modelos de
distribución espacial (SDM por sus siglas en ingles) que permiten a
partir de unos datos de ocurrencia generar modelos de distribución
espacial. A continuación les presento una serie de pasos que permitirá
desarrollar modelos de distribución espacial. El presente ejercicio está
basado en el Blog \emph{Spatial Data Science} de
\href{https://rspatial.org/raster/sdm/index.html}{Robert Hijmans}

\subsection{Preparación de datos}\label{preparacion-de-datos}

Una de las fases primordiales para el desarrollo de un \emph{SDM} son
los datos de partida. Para poder implementar un modelo necesitamos
obtener datos de ocurrencia (coordenadas espaciales), donde la especie
ha sido catalogada. Existen múltiples bases de datos con
\emph{información de ocurrencia}, pero también los investigadores
podrían tener puntos de ocurrencia levantados por ellos mismos. Una vez
que se obtienen los datos, es necesario verificar que las coordenadas no
sean erradas o que existan problemas con estos datos de ocurrencia,
necesitamos \emph{limpiar los datos}.

\subsubsection{Importando datos de
ocurrencia}\label{importando-datos-de-ocurrencia}

Cuando el investigador cuenta con datos de ocurrencia levantados por él
o que ha levantado de diversas fuentes, necesitamos leerlo en R. Vamos a
utilizar datos de un murciélago vampiro que ocurre en las áreas
tropicales y que ha sido bastante conocido por atacar al ganado. El
nombre científico de esta especie es \emph{Desmodus rotundus}.

\begin{Shaded}
\begin{Highlighting}[]
\KeywordTok{library}\NormalTok{(readxl)}

\NormalTok{desmod <-}\StringTok{ }\KeywordTok{read_excel}\NormalTok{(}\StringTok{"Ocurrencia Desmodus.xlsx"}\NormalTok{)}
\end{Highlighting}
\end{Shaded}

Ahora verificamos que los datos hayan sido leídos adecuadamente.

\begin{Shaded}
\begin{Highlighting}[]
\KeywordTok{str}\NormalTok{(desmod)}
\end{Highlighting}
\end{Shaded}

\begin{verbatim}
## Classes 'tbl_df', 'tbl' and 'data.frame':    251 obs. of  3 variables:
##  $ lat: num  -4.63 -4.56 -4.56 -4.51 -4.48 ...
##  $ lon: num  -79.5 -79.5 -79.5 -79.5 -80.4 ...
##  $ ID : num  170591 170114 170114 170921 168803 ...
\end{verbatim}

La función \emph{str} nos permite ver las propiedades de la matriz de
datos leída. Como vemos tenemos 251 registros y tres variables, latitud,
longitud y un identificador.

Ahora es posible que estemos iniciando el trabajo de modelado por lo que
no disponemos de datos de ocurrencias, podríamos cargar los datos que
han sido depositados en el \href{https://www.gbif.org/}{Global
Biodiversity Inventory Facility (GBIF)} usando la función \emph{gbif}
del paquete \textbf{dismo}. Si es la primera vez que usará este paquete
seguramente tendrá que descargarlo; use la siguiente línea de código
para hacerlo: \texttt{install.packages(“dismo”)}

\begin{Shaded}
\begin{Highlighting}[]
\KeywordTok{library}\NormalTok{(dismo)}
\end{Highlighting}
\end{Shaded}

\begin{verbatim}
## Loading required package: raster
\end{verbatim}

\begin{verbatim}
## Loading required package: sp
\end{verbatim}

\begin{Shaded}
\begin{Highlighting}[]
\CommentTok{# Usar las dos líneas de código a continuación para extraer }
\CommentTok{#los datos de GBIF y guardarlos en su computador}

\CommentTok{#La siguiente línea extrae los datos }
\CommentTok{# desGbif <- gbif("Desmodus", "rotundus", geo = TRUE) }

\CommentTok{#La siguiente línea graba los datos en su disco duro}
\CommentTok{# write.csv(desGbif, "desGbif.csv")}

\CommentTok{#La siguiente vez que realice el trabajo, ejecute solo leer los datos del disco duro}

\NormalTok{desGbif <-}\StringTok{ }\KeywordTok{read.csv}\NormalTok{(}\StringTok{"desGbif.csv"}\NormalTok{)}
\end{Highlighting}
\end{Shaded}

Pueden ver otros argumentos posibles a usarse con la función GBIF, se
podría usar un shape o un polígono para obtener los datos de una región
que no interese. A continuación generamos un mapa con el fin de
verificar que los puntos tengan una ubicación correcta.

\begin{quote}
\textbf{Nota:} se puede utilizar la función \emph{getData} para obtener
un Shapefile de un lugar que nos interese por ejemplo Ecuador o la
función \emph{shapefile} si queremos cargar un Shape que ya tengamos en
el computador. Estas dos funciones se encuentran en el paquete
\textbf{raster}
\end{quote}

\begin{Shaded}
\begin{Highlighting}[]
\KeywordTok{library}\NormalTok{(maptools)}
\KeywordTok{data}\NormalTok{(wrld_simpl)}
\CommentTok{#graficamos un mapa del mundo}
\KeywordTok{plot}\NormalTok{(wrld_simpl, }\DataTypeTok{xlim=}\KeywordTok{c}\NormalTok{(-}\DecValTok{100}\NormalTok{,}\DecValTok{40}\NormalTok{), }\DataTypeTok{ylim=}\KeywordTok{c}\NormalTok{(-}\DecValTok{60}\NormalTok{,}\DecValTok{40}\NormalTok{), }\DataTypeTok{axes=}\OtherTok{TRUE}\NormalTok{, }\DataTypeTok{col=}\StringTok{"light yellow"}\NormalTok{)}
\KeywordTok{box}\NormalTok{()}
\CommentTok{# Adicionamos los puntos de ocurrencia}
\KeywordTok{points}\NormalTok{(desGbif$lon, desGbif$lat, }\DataTypeTok{bg=}\StringTok{'orange'}\NormalTok{, }\DataTypeTok{pch=}\DecValTok{21}\NormalTok{, }\DataTypeTok{cex=}\FloatTok{0.75}\NormalTok{)}
\CommentTok{# plot points again to add a border, for better visibility}
\KeywordTok{points}\NormalTok{(desmod$lon, desmod$lat, }\DataTypeTok{col=}\StringTok{'red'}\NormalTok{, }\DataTypeTok{cex=}\FloatTok{0.75}\NormalTok{)}
\end{Highlighting}
\end{Shaded}

\includegraphics{index_files/figure-latex/unnamed-chunk-4-1.pdf}

\subsubsection{Limpieza de datos}\label{limpieza-de-datos}

Un paso fundamental para poder construir los modelos es realizar una
limpieza de datos, sobre todo si los datos provienen directamente de
bases de datos como la de GBIF. Como podemos observar en el gráfico
anterior existe un punto en medio del Atlántico. Es necesario limpiar
esta serie de inconsistencias.

Un punto importante es evaluar si no existen datos con coordenadas
iguales, ver si existen datos duplicados de latitud y longitud

\begin{Shaded}
\begin{Highlighting}[]
\NormalTok{dupl <-}\StringTok{ }\KeywordTok{duplicated}\NormalTok{(desGbif[, }\KeywordTok{c}\NormalTok{(}\StringTok{'lon'}\NormalTok{, }\StringTok{'lat'}\NormalTok{)])}
\CommentTok{#Cuantos datos duplicados tenemos}
\KeywordTok{sum}\NormalTok{(dupl)}
\end{Highlighting}
\end{Shaded}

\begin{verbatim}
## [1] 16955
\end{verbatim}

\begin{Shaded}
\begin{Highlighting}[]
\CommentTok{#Tenemos 16951 datos duplicados que deberíamos excluir}
\CommentTok{#Usamos este vector para eliminar los duplicados}
\NormalTok{desGbifCd <-}\StringTok{ }\NormalTok{desGbif[!dupl,]}
\end{Highlighting}
\end{Shaded}

Ahora necesitamos eliminar todos los registros que no tienen datos de
latitud y longitud.

\begin{Shaded}
\begin{Highlighting}[]
\NormalTok{lolNA <-}\StringTok{ }\KeywordTok{which}\NormalTok{(}\KeywordTok{is.na}\NormalTok{(desGbifCd$lon)&}\KeywordTok{is.na}\NormalTok{(desGbifCd$lat))}
\NormalTok{desGbifCd <-}\StringTok{ }\NormalTok{desGbifCd[-lolNA,]}
\end{Highlighting}
\end{Shaded}

Hemos realizado algunos procesos de limpieza de datos, finalmente
podemos hacer una corrección cruzada, usaremos los datos de los países y
veremos si los datos corresponden a los países.

\begin{Shaded}
\begin{Highlighting}[]
\KeywordTok{library}\NormalTok{(sp)}
\NormalTok{csDes <-}\StringTok{ }\NormalTok{desGbifCd}
\NormalTok{csDes$country <-}\StringTok{ }\KeywordTok{droplevels}\NormalTok{(csDes$country, }\DataTypeTok{exclude =} \DecValTok{0}\NormalTok{)}
\KeywordTok{coordinates}\NormalTok{(csDes) <-}\StringTok{  }\ErrorTok{~}\NormalTok{lon+lat}
\KeywordTok{crs}\NormalTok{(csDes) <-}\StringTok{ }\KeywordTok{crs}\NormalTok{(wrld_simpl)}
\KeywordTok{class}\NormalTok{(csDes)}
\end{Highlighting}
\end{Shaded}

\begin{verbatim}
## [1] "SpatialPointsDataFrame"
## attr(,"package")
## [1] "sp"
\end{verbatim}

\begin{Shaded}
\begin{Highlighting}[]
\NormalTok{ovr <-}\StringTok{ }\KeywordTok{over}\NormalTok{(csDes, wrld_simpl)}
\KeywordTok{head}\NormalTok{(ovr)}
\end{Highlighting}
\end{Shaded}

\begin{verbatim}
##   FIPS ISO2 ISO3  UN       NAME   AREA   POP2005 REGION SUBREGION      LON
## 1   MX   MX  MEX 484     Mexico 190869 104266392     19        13 -102.535
## 2   CO   CO  COL 170   Colombia 103870   4494579     19         5  -73.076
## 3   MX   MX  MEX 484     Mexico 190869 104266392     19        13 -102.535
## 4   CS   CR  CRI 188 Costa Rica   5106   4327228     19        13  -83.946
## 5   PE   PE  PER 604       Peru 128000  27274266     19         5  -75.552
## 6   MX   MX  MEX 484     Mexico 190869 104266392     19        13 -102.535
##      LAT
## 1 23.951
## 2  3.900
## 3 23.951
## 4  9.971
## 5 -9.326
## 6 23.951
\end{verbatim}

Vemos algunos datos que tienen NA como país, estos corresponden a los
datos que se encuentran en el océano y que no corresponden a ocurrencia
de la especie.

\begin{Shaded}
\begin{Highlighting}[]
\NormalTok{cntr <-}\StringTok{ }\NormalTok{ovr$NAME}

\CommentTok{#which nos permite ver la ubicación de los puntos}
\NormalTok{i <-}\StringTok{ }\KeywordTok{which}\NormalTok{(}\KeywordTok{is.na}\NormalTok{(cntr))}
\KeywordTok{length}\NormalTok{(i) }\CommentTok{#Cuantos datos sin país}
\end{Highlighting}
\end{Shaded}

\begin{verbatim}
## [1] 117
\end{verbatim}

\begin{Shaded}
\begin{Highlighting}[]
\CommentTok{#Definimos las ocurrencias que coinciden}
\NormalTok{j <-}\StringTok{ }\KeywordTok{which}\NormalTok{(}\KeywordTok{as.character}\NormalTok{(cntr)==}\KeywordTok{as.character}\NormalTok{(csDes$country))}

\CommentTok{#vemos a que corresponden los datos que no coinciden}
\KeywordTok{cbind.data.frame}\NormalTok{(cntr, csDes$country)[-}\KeywordTok{c}\NormalTok{(j,i),]}
\end{Highlighting}
\end{Shaded}

\begin{verbatim}
##               cntr csDes$country
## 39       Venezuela      Colombia
## 124      Venezuela      Colombia
## 250       Paraguay     Argentina
## 269         Brazil      Suriname
## 296     Costa Rica        Panama
## 320         Guyana      Suriname
## 336       Suriname French Guiana
## 350      Venezuela      Colombia
## 355         Guyana      Suriname
## 358         Guyana      Suriname
## 362         Guyana      Suriname
## 515         Mexico     Guatemala
## 694      Guatemala        Mexico
## 885       Suriname        Guyana
## 886       Suriname        Guyana
## 887       Suriname        Guyana
## 941         Belize        Mexico
## 951      Guatemala        Mexico
## 974      Guatemala        Mexico
## 1219        Belize        Mexico
## 1226     Guatemala        Mexico
## 1335        Brazil          Peru
## 1361          Peru       Ecuador
## 1417        Mexico        Belize
## 1552      Honduras   El Salvador
## 2012     Venezuela      Colombia
## 2133      Colombia     Venezuela
## 2922      Viet Nam      Colombia
## 2945     Venezuela      Honduras
## 2965       Bolivia        Brazil
## 2998      Paraguay     Argentina
## 3112 United States        Mexico
## 3172       Bolivia     Argentina
## 3214      Paraguay     Argentina
\end{verbatim}

La mayoría de los datos corresponden a efectos de la precisión de los
mapas, como vemos los errores se dan entre países vecinos. Si tuviésemos
pocos puntos deberíamos verificar y buscar el error para no perder estos
datos, sin embargo, como tenemos suficientes datos vamos a eliminar
estos puntos. Eliminemos los datos que caen en el mar (NA) y los que
tienen coordenadas erradas y veamos cómo quedan nuestros datos.

\begin{Shaded}
\begin{Highlighting}[]
\CommentTok{#removemos los datos que con ubicación NA}
\NormalTok{desmodF <-}\StringTok{ }\NormalTok{desGbifCd[-i,]}
\end{Highlighting}
\end{Shaded}

Finalmente, hemos eliminado las ocurrencias con problemas. Pero en el
mapa aún tenemos puntos fuera del neotrópico, optaremos por eliminar
estos puntos ya que se supone esta especie está restringida al
neotrópico.

\begin{Shaded}
\begin{Highlighting}[]
\CommentTok{#Ubicamos el país}
\NormalTok{us <-}\StringTok{ }\KeywordTok{which}\NormalTok{(desmodF$country==}\StringTok{"United States"}\NormalTok{) }

\CommentTok{#Registros sobre 100 grados de longitud}
\NormalTok{sa <-}\StringTok{ }\KeywordTok{which}\NormalTok{(desmodF$lon>}\DecValTok{80}\NormalTok{)}

\NormalTok{desmodF <-}\StringTok{ }\NormalTok{desmodF[-}\KeywordTok{c}\NormalTok{(us,sa), ]}
\end{Highlighting}
\end{Shaded}

Tenemos varios datos que hemos ido limpiando a lo largo de este
ejercicio hasta llegar a tener unas coordenadas limpias. Sin embargo,
puesto que realizaremos modelos predictivos con estas ocurrencias, es
necesario reducir los posibles efectos de colinearidad espacial. Si
tenemos dos registros de ocurrencia cercanos, por debajo de 500 metros
(que es la resolución de las variables explicativas que veremos más
adelante), es necesario quedarnos con uno solo de estos registros. A
continuación vamos a eliminar los registros que se encuentran dentro de
una misma celda.

\begin{Shaded}
\begin{Highlighting}[]
\KeywordTok{library}\NormalTok{(dismo)}
\CommentTok{# Obtenemos los datos del paquete dismo}
\NormalTok{files <-}\StringTok{ }\KeywordTok{list.files}\NormalTok{(}\DataTypeTok{path=}\KeywordTok{paste}\NormalTok{(}\KeywordTok{system.file}\NormalTok{(}\DataTypeTok{package=}\StringTok{"dismo"}\NormalTok{), }\StringTok{'/ex'}\NormalTok{,}
                       \DataTypeTok{sep=}\StringTok{''}\NormalTok{),  }\DataTypeTok{pattern=}\StringTok{'grd'}\NormalTok{,  }\DataTypeTok{full.names=}\OtherTok{TRUE} \NormalTok{)}

\CommentTok{#Usamos el primer raster para obtener la mascara }
\NormalTok{mask <-}\StringTok{ }\KeywordTok{raster}\NormalTok{(files[}\DecValTok{1}\NormalTok{])}
\NormalTok{maskU <-}\StringTok{ }\NormalTok{mask}
\KeywordTok{values}\NormalTok{(maskU) <-}\StringTok{ }\DecValTok{1}\NormalTok{:}\KeywordTok{ncell}\NormalTok{(mask)}

\CommentTok{#convertimos en un objeto tipo SpatialPointDataFrame}
\NormalTok{desRep <-}\StringTok{ }\NormalTok{desmodF}
\KeywordTok{coordinates}\NormalTok{(desRep) <-}\StringTok{ }\ErrorTok{~}\NormalTok{lon+lat}
\KeywordTok{projection}\NormalTok{(desRep) <-}\StringTok{ }\KeywordTok{CRS}\NormalTok{(}\StringTok{'+proj=longlat +datum=WGS84'}\NormalTok{)}

\CommentTok{#Usamos extrac para obtener el valor de cada punto de ocurrencia}
\NormalTok{xp <-}\StringTok{ }\KeywordTok{extract}\NormalTok{(mask, desRep)}
\KeywordTok{length}\NormalTok{(xp) }\CommentTok{#la cantidad de puntos que tenemos }
\end{Highlighting}
\end{Shaded}

\begin{verbatim}
## [1] 3220
\end{verbatim}

\begin{Shaded}
\begin{Highlighting}[]
\KeywordTok{length}\NormalTok{(}\KeywordTok{unique}\NormalTok{(xp)) }\CommentTok{#la cantidad de puntos con una ubicación mayor a }
\end{Highlighting}
\end{Shaded}

\begin{verbatim}
## [1] 170
\end{verbatim}

\begin{Shaded}
\begin{Highlighting}[]
                   \CommentTok{#500 metros}
\CommentTok{#definimos la ubicación de los duplicados}
\NormalTok{dupDes <-}\StringTok{ }\KeywordTok{which}\NormalTok{(}\KeywordTok{duplicated}\NormalTok{(xp))}

\CommentTok{#usamos este vector para eliminar duplicados}
\NormalTok{desmodFD <-}\StringTok{ }\NormalTok{desmodF[-dupDes,]}

\CommentTok{#graficamos nuevamente}
\KeywordTok{plot}\NormalTok{(wrld_simpl, }\DataTypeTok{xlim=}\KeywordTok{c}\NormalTok{(-}\DecValTok{100}\NormalTok{,}\DecValTok{50}\NormalTok{), }\DataTypeTok{ylim=}\KeywordTok{c}\NormalTok{(-}\DecValTok{50}\NormalTok{,}\DecValTok{30}\NormalTok{), }\DataTypeTok{axes=}\OtherTok{TRUE}\NormalTok{, }\DataTypeTok{col=}\StringTok{"light yellow"}\NormalTok{)}
\KeywordTok{box}\NormalTok{()}
\CommentTok{# incluimos los puntos}
\KeywordTok{points}\NormalTok{(desmodFD$lon, desmodFD$lat, }\DataTypeTok{bg=}\StringTok{'orange'}\NormalTok{, }\DataTypeTok{pch=}\DecValTok{21}\NormalTok{, }\DataTypeTok{cex=}\FloatTok{0.75}\NormalTok{)}
\end{Highlighting}
\end{Shaded}

\includegraphics{index_files/figure-latex/unnamed-chunk-11-1.pdf}

Ahora tenemos unos datos limpios y listos para ser usados en nuestro
modelo. Aunque a nosotros tomamos poco tiempo para la revisión,
deberíamos disponer de más tiempo para revisar los datos y realizar una
adecuada limpieza. Durante este ejercicio hemos tomado varias decisiones
que deben ser mejor evaluadas.

\subsection{Datos de ausencia y de
fondo}\label{datos-de-ausencia-y-de-fondo}

Los algoritmos para realizar modelos de distribución de especies, como
Bioclim y Domain, usan datos de ``presencia'' para realizar el modelado
y estos no necesitan información de ausencias. Estos modelos son
conocidos como modelos de envuelta ambiental, porque justamente se basan
en las características ambientales que tiene cada uno de los datos de
presencia. Aunque estos modelos fueron bastante utilizados, hoy en día
han perdido importancia. Actualmente los métodos de modelado de
distribución de especies usan datos de ``\emph{ausencia}'' o datos de
``\emph{fondo}''.

Los datos de ausencia se refiere a los sitios donde aunque hemos buscado
la especie, efectivamente no la hemos encontrado. Si solo tenemos datos
de presencia, podemos usar un método que necesite datos de ausencia,
sustituyendo los datos de ausencia con datos de fondo o
pseudo-ausencias. En este caso denominamos pseudo-ausencias o datos de
fondo porque vamos a decir que en un sitio determinado no existe la
especie aunque realmente no hemos realizado un muestreo para determinar
que no está. Aunque, los datos de fondo y pseudo-ausencias hacen
referencia a lo mismo, la forma de obtener estos datos es diferente.
Preferimos el concepto de fondo porque requiere menos suposiciones y
tiene algunos métodos estadísticos coherentes para lidiar con la
``superposición'' entre la presencia y los puntos de fondo (por ejemplo,
Ward et al. 2009; Phillips y Elith, 2011).

El paquete \textbf{dismo} tiene una función para muestrear puntos
aleatorios (datos de fondo) de un área de estudio. Puede usar una
``máscara'' para excluir el área no terrestre que no tiene datos (NA).
También se puede usar una ``extensión'' para restringir la generación de
puntos a un área más pequeña.

A continuación cargaremos unos datos raster (más adelante hablaremos de
ello) que usaremos como mascara, básicamente generaremos unos puntos
aleatorios donde el raster tenga datos. Usaremos la función
\emph{randomPoints} para generar los puntos aleatorios en esta mascara,
además generaremos un cuadrante más pequeño donde se generen los datos
de fondo, usaremos el argumento ``ext'' de la misma función.

\begin{Shaded}
\begin{Highlighting}[]
\CommentTok{# Seleccionamos 500 puntos aleatorios}
\CommentTok{#aseguramos tener los mismos datos aleatorios}
\KeywordTok{set.seed}\NormalTok{(}\DecValTok{1963}\NormalTok{)}
\CommentTok{#Generamos los datos aleatorios}
\NormalTok{bg <-}\StringTok{ }\KeywordTok{randomPoints}\NormalTok{(mask, }\DecValTok{500} \NormalTok{) }

\CommentTok{#generamos un cuadrante donde generar los datos aleatorios}
\NormalTok{e <-}\StringTok{ }\KeywordTok{extent}\NormalTok{(-}\DecValTok{85}\NormalTok{, -}\DecValTok{70}\NormalTok{, -}\DecValTok{10}\NormalTok{, }\DecValTok{5}\NormalTok{)}
\NormalTok{bg2 <-}\StringTok{ }\KeywordTok{randomPoints}\NormalTok{(mask, }\DecValTok{50}\NormalTok{, }\DataTypeTok{ext=}\NormalTok{e)}

\CommentTok{# set up the plotting area for two maps}
\KeywordTok{par}\NormalTok{(}\DataTypeTok{mfrow=}\KeywordTok{c}\NormalTok{(}\DecValTok{1}\NormalTok{,}\DecValTok{2}\NormalTok{), }\DataTypeTok{mar=}\KeywordTok{c}\NormalTok{(}\DecValTok{3}\NormalTok{,}\DecValTok{3}\NormalTok{,}\DecValTok{1}\NormalTok{,}\DecValTok{1}\NormalTok{))}
\KeywordTok{plot}\NormalTok{(!}\KeywordTok{is.na}\NormalTok{(mask), }\DataTypeTok{legend=}\OtherTok{FALSE}\NormalTok{)}
\KeywordTok{points}\NormalTok{(bg, }\DataTypeTok{cex=}\FloatTok{0.3}\NormalTok{, }\DataTypeTok{pch=}\DecValTok{21}\NormalTok{)}
\CommentTok{# now we repeat the sampling, but limit}
\CommentTok{# the area of sampling using a spatial extent}
\KeywordTok{plot}\NormalTok{(!}\KeywordTok{is.na}\NormalTok{(mask), }\DataTypeTok{legend=}\OtherTok{FALSE}\NormalTok{)}
\KeywordTok{plot}\NormalTok{(e, }\DataTypeTok{add=}\OtherTok{TRUE}\NormalTok{, }\DataTypeTok{col=}\StringTok{'red'}\NormalTok{)}
\KeywordTok{points}\NormalTok{(bg2, }\DataTypeTok{cex=}\FloatTok{0.3}\NormalTok{, }\DataTypeTok{pch=}\DecValTok{21}\NormalTok{)}
\end{Highlighting}
\end{Shaded}

\includegraphics{index_files/figure-latex/unnamed-chunk-12-1.pdf} Hay
varios enfoques que uno podría usar para muestrear puntos de
``pseudo-ausencia'', es decir, puntos de un área más restringida que los
datos de ``fondo''. VanDerWal et al. (2009) proponen realizar un
muestreo en un radio alrededor de los puntos de presencia. Vamos a usar
los datos de \emph{Desmodus rotundus} que habíamos limpiado previamente
para generar los radios.

Lo primero que haremos es cambiar los datos que se encuentran como un
arreglo de datos (data.frame) y los convertiremos en datos espaciales
(SpatialPointsDataFrame)

\begin{Shaded}
\begin{Highlighting}[]
\CommentTok{# Generamos círculos con un radio de 50 km}
\CommentTok{# usamos el objeto tipo SpatialPointDataFrame}
\NormalTok{x <-}\StringTok{ }\KeywordTok{circles}\NormalTok{(desRep, }\DataTypeTok{d=}\DecValTok{50000}\NormalTok{, }\DataTypeTok{lonlat=}\OtherTok{TRUE}\NormalTok{)}
\NormalTok{## Lo convertimos en un polígono}
\NormalTok{pol <-}\StringTok{ }\KeywordTok{polygons}\NormalTok{(x)}
\end{Highlighting}
\end{Shaded}

Lo función \emph{polygons} elimina las zonas en las cuales los círculos
se superponen. Finalmente, tomamos una muestra aleatoria de puntos
dentro de los polígonos que hemos creado. Solo queremos un punto por
celda de cuadrícula.

\begin{Shaded}
\begin{Highlighting}[]
\CommentTok{# Muestreamos aleatoriamente dentro del polígono}
\CommentTok{# extraemos 250 puntos}
\KeywordTok{set.seed}\NormalTok{(}\DecValTok{34}\NormalTok{)}
\NormalTok{samp1 <-}\StringTok{ }\KeywordTok{spsample}\NormalTok{(pol, }\DecValTok{250}\NormalTok{, }\DataTypeTok{type=}\StringTok{'random'}\NormalTok{, }\DataTypeTok{iter=}\DecValTok{25}\NormalTok{)}

\CommentTok{# Extraemos el valor para cada punto de la mascara}
\NormalTok{cells <-}\StringTok{ }\KeywordTok{cellFromXY}\NormalTok{(mask, samp1)}
\KeywordTok{length}\NormalTok{(cells) }\CommentTok{#tenemos 250 puntos los que generamos}
\end{Highlighting}
\end{Shaded}

\begin{verbatim}
## [1] 250
\end{verbatim}

\begin{Shaded}
\begin{Highlighting}[]
\CommentTok{#Eliminamos puntos repetidos}
\CommentTok{#Los puntos que tengan el mismo valor serán aquellos que }
\CommentTok{#están dentro de una misma celda}
\NormalTok{cells <-}\StringTok{ }\KeywordTok{unique}\NormalTok{(cells)}
\KeywordTok{length}\NormalTok{(cells) }\CommentTok{#17 puntos estaban en una misma celda }
\end{Highlighting}
\end{Shaded}

\begin{verbatim}
## [1] 227
\end{verbatim}

\begin{Shaded}
\begin{Highlighting}[]
              \CommentTok{#los eliminamos y nos quedan 233 puntos }

\CommentTok{#obtenemos las coordenadas de esos puntos}
\NormalTok{xy <-}\StringTok{ }\KeywordTok{xyFromCell}\NormalTok{(mask, cells) }

\NormalTok{##Graficamos}
\KeywordTok{plot}\NormalTok{(pol, }\DataTypeTok{axes=}\OtherTok{TRUE}\NormalTok{)}
\KeywordTok{points}\NormalTok{(xy, }\DataTypeTok{cex=}\FloatTok{0.4}\NormalTok{, }\DataTypeTok{pch=}\DecValTok{3}\NormalTok{, }\DataTypeTok{col=}\StringTok{'blue'}\NormalTok{)}
\end{Highlighting}
\end{Shaded}

\includegraphics{index_files/figure-latex/unnamed-chunk-14-1.pdf}

Algunos puntos aleatorios han quedado fuera del polígono que creamos.
Esto pueda pasar ya que al generar coordenadas a partir de las celdas, R
nos da el centroide, y esto puede ocasionar que algunos puntos se salgan
de la máscara. Podemos seleccionar solo aquellos puntos que se
encuentran dentro de la máscara, nuevamente usaremos la función
\emph{over}.

\begin{Shaded}
\begin{Highlighting}[]
\NormalTok{spxy <-}\StringTok{ }\KeywordTok{SpatialPoints}\NormalTok{(}\KeywordTok{na.omit}\NormalTok{(xy), }\DataTypeTok{proj4string=}\KeywordTok{CRS}\NormalTok{(}\StringTok{'+proj=longlat +datum=WGS84'}\NormalTok{))}
\NormalTok{o <-}\StringTok{ }\KeywordTok{over}\NormalTok{(spxy, }\KeywordTok{geometry}\NormalTok{(x))}
\NormalTok{xyInside <-}\StringTok{ }\NormalTok{xy[!}\KeywordTok{is.na}\NormalTok{(o), ]}
\end{Highlighting}
\end{Shaded}

\subsection{Datos ambientales}\label{datos-ambientales}

\subsubsection{Datos raster}\label{datos-raster}

Los modelos de distribución de especies necesitan, por un lado, los
puntos de presencia que acabamos de limpiar y los datos de ausencia que
generamos, pero además, necesitamos variables que expliquen esa
distribución (variables explicativas). Generalmente las variables
explicativas son variables climáticas, geomorfológicas entre otras
(clima, suelo, terreno, vegetación, uso del suelo, etc.)

Las variables explicativas generalmente se organizan como archivos de
tipo raster (cuadrícula), de tal forma que tendremos tantos raster como
variable explicativa a ser usada. Estos datos generalmente se almacenan
en archivos en algún tipo de formato SIG. Se pueden utilizar casi todos
los formatos relevantes (incluida la cuadrícula ESRI, geoTiff, netCDF,
IDRISI). Los archivos ASCII pueden ser usados aunque estos tienden a
reducir considerablemente la velocidad de procesamiento. Para cualquier
estudio en particular, todas las capas deben tener la misma extensión
espacial, resolución, origen y proyección. Si es necesario, use
funciones como \emph{crop}, \emph{extend}, \emph{aggregate},
\emph{resample} y \emph{projectRaster} desde el paquete \textbf{raster}.

El paquete \textbf{dismo} trae cargadas varias variables provenientes de
la base de datos de WorldClim (Hijmans et al., 2004) y los datos de
biomas terrestres de la WWF (Olsen et al., 2001). A continuación
desarrollemos los pasos para extraer los datos del paquete dismo y
empaquetarlos en un stack que nos permita procesar en conjunto todas las
variables.

\begin{Shaded}
\begin{Highlighting}[]
\CommentTok{#Definimos el directorio donde están los archivos}
\NormalTok{path <-}\StringTok{ }\KeywordTok{file.path}\NormalTok{(}\KeywordTok{system.file}\NormalTok{(}\DataTypeTok{package=}\StringTok{"dismo"}\NormalTok{), }\StringTok{'ex'}\NormalTok{)}
\CommentTok{#obtenemos un listado de las variables tipo grd (grid) que están en dismo}
\CommentTok{#grd$ nos permite obtener todos las variables grid}
\KeywordTok{library}\NormalTok{(dismo)}
\NormalTok{files <-}\StringTok{ }\KeywordTok{list.files}\NormalTok{(path, }\DataTypeTok{pattern=}\StringTok{'grd$'}\NormalTok{, }\DataTypeTok{full.names=}\OtherTok{TRUE} \NormalTok{)}
\NormalTok{files}
\end{Highlighting}
\end{Shaded}

\begin{verbatim}
## [1] "C:/Users/ciesp/Documents/R/win-library/3.5/dismo/ex/bio1.grd" 
## [2] "C:/Users/ciesp/Documents/R/win-library/3.5/dismo/ex/bio12.grd"
## [3] "C:/Users/ciesp/Documents/R/win-library/3.5/dismo/ex/bio16.grd"
## [4] "C:/Users/ciesp/Documents/R/win-library/3.5/dismo/ex/bio17.grd"
## [5] "C:/Users/ciesp/Documents/R/win-library/3.5/dismo/ex/bio5.grd" 
## [6] "C:/Users/ciesp/Documents/R/win-library/3.5/dismo/ex/bio6.grd" 
## [7] "C:/Users/ciesp/Documents/R/win-library/3.5/dismo/ex/bio7.grd" 
## [8] "C:/Users/ciesp/Documents/R/win-library/3.5/dismo/ex/bio8.grd" 
## [9] "C:/Users/ciesp/Documents/R/win-library/3.5/dismo/ex/biome.grd"
\end{verbatim}

\begin{Shaded}
\begin{Highlighting}[]
\CommentTok{#empaquetamos las variables en un stack}
\NormalTok{predictors <-}\StringTok{ }\KeywordTok{stack}\NormalTok{(files)}
\NormalTok{predictors}
\end{Highlighting}
\end{Shaded}

\begin{verbatim}
## class      : RasterStack 
## dimensions : 192, 186, 35712, 9  (nrow, ncol, ncell, nlayers)
## resolution : 0.5, 0.5  (x, y)
## extent     : -125, -32, -56, 40  (xmin, xmax, ymin, ymax)
## crs        : +proj=longlat +datum=WGS84 +ellps=WGS84 +towgs84=0,0,0 
## names      : bio1, bio12, bio16, bio17, bio5, bio6, bio7, bio8, biome 
## min values :  -23,     0,     0,     0,   61, -212,   60,  -66,     1 
## max values :  289,  7682,  2458,  1496,  422,  242,  461,  323,    14
\end{verbatim}

\begin{Shaded}
\begin{Highlighting}[]
\KeywordTok{names}\NormalTok{(predictors)}
\end{Highlighting}
\end{Shaded}

\begin{verbatim}
## [1] "bio1"  "bio12" "bio16" "bio17" "bio5"  "bio6"  "bio7"  "bio8"  "biome"
\end{verbatim}

\begin{Shaded}
\begin{Highlighting}[]
\KeywordTok{plot}\NormalTok{(predictors)}
\end{Highlighting}
\end{Shaded}

\includegraphics{index_files/figure-latex/unnamed-chunk-16-1.pdf} Vamos
a graficar una de las variables y sobreponer los datos geopolíticos e
incluir los datos de Desmodus que hemos limpiado.

\begin{Shaded}
\begin{Highlighting}[]
\KeywordTok{library}\NormalTok{(maptools)}
\CommentTok{#generamos un objeto tipo SpatialPoint}
\NormalTok{desFD <-}\StringTok{ }\NormalTok{desmodFD}
\KeywordTok{coordinates}\NormalTok{(desFD) <-}\StringTok{ }\ErrorTok{~}\NormalTok{lon+lat}
\KeywordTok{projection}\NormalTok{(desFD) <-}\StringTok{ }\KeywordTok{CRS}\NormalTok{(}\StringTok{'+proj=longlat +datum=WGS84'}\NormalTok{)}

\CommentTok{# Graficamos la primera variable climática}
\KeywordTok{plot}\NormalTok{(predictors, }\DecValTok{1}\NormalTok{)}
\CommentTok{# Adicionamos los límites políticos}
\CommentTok{#usamos add=TRUE para que se sobreponga sobre el gráfico anterior}
\KeywordTok{plot}\NormalTok{(wrld_simpl, }\DataTypeTok{add=}\OtherTok{TRUE}\NormalTok{)}
\CommentTok{#finalmente adicionamos los datos de Desmodus}
\KeywordTok{points}\NormalTok{(desFD, }\DataTypeTok{col=}\StringTok{"darkred"}\NormalTok{, }\DataTypeTok{pch=}\DecValTok{3}\NormalTok{, }\DataTypeTok{cex=}\FloatTok{0.3}\NormalTok{)}
\end{Highlighting}
\end{Shaded}

\includegraphics{index_files/figure-latex/unnamed-chunk-17-1.pdf}

Obtener las variables explicativas es un paso muy importante, sobre todo
si nuestro enfoque es predictivo y no explicativo. El modelado de
distribución geográfica no busca comprender el efecto de las variables
explicativas, sino más bien nos interesa tener capacidad para predecir
la distribución. De esta forma, es vital tener variables que permitan
realizar la predicción, para lo cual es importante conocer la ecología
de la especie. Variables que son importantes para una especie pueden ser
irrelevantes para otras, ej. una variable de suelo puede ser muy
importante para una planta, pero irrelevante para un ave. En el proceso
de modelado necesitamos contar con variables relevantes para la especie
que estamos modelando.

\subsubsection{Extrayendo los datos
ambientales}\label{extrayendo-los-datos-ambientales}

Ahora que tenemos tanto nuestras variables de presencia-ausencia como
nuestras variables explicativas, el siguiente paso es extraer los
valores de las variables explicativas. Cada punto de ocurrencia y
ausencia le corresponde un valor de cada variable explicativa. Usaremos
la función \emph{extract} del paquete \textbf{raster} para obtener los
valores de las variables contenidas en el stack, usaremos tanto los
datos de ocurrencia que limpiamos como los datos de fondo que generamos.

\begin{Shaded}
\begin{Highlighting}[]
\CommentTok{#extraemos los valores}
\NormalTok{presvals <-}\StringTok{ }\KeywordTok{extract}\NormalTok{(predictors, desFD) }\CommentTok{#presencias}
\NormalTok{absvals <-}\StringTok{ }\KeywordTok{extract}\NormalTok{(predictors, bg) }\CommentTok{#datos de fondo}
\NormalTok{seabsvals <-}\StringTok{ }\KeywordTok{extract}\NormalTok{(predictors, xyInside) }\CommentTok{#pseudo-ausencias}

\CommentTok{#generamos una variable de presencia ausencia}
\NormalTok{pb <-}\StringTok{ }\KeywordTok{c}\NormalTok{(}\KeywordTok{rep}\NormalTok{(}\DecValTok{1}\NormalTok{, }\KeywordTok{nrow}\NormalTok{(presvals)), }\KeywordTok{rep}\NormalTok{(}\DecValTok{0}\NormalTok{, }\KeywordTok{nrow}\NormalTok{(absvals)))}

\CommentTok{#unimos los datos de presencia-ausencia y la juntamos con la variable}
\CommentTok{#que acabamos de generar}
\NormalTok{sdmdata <-}\StringTok{ }\KeywordTok{data.frame}\NormalTok{(}\KeywordTok{cbind}\NormalTok{(pb, }\KeywordTok{rbind}\NormalTok{(presvals, absvals)))}
\CommentTok{#convertimos en factor la variable biome}
\NormalTok{sdmdata[,}\StringTok{'biome'}\NormalTok{] =}\StringTok{ }\KeywordTok{as.factor}\NormalTok{(sdmdata[,}\StringTok{'biome'}\NormalTok{])}
\KeywordTok{head}\NormalTok{(sdmdata)}
\end{Highlighting}
\end{Shaded}

\begin{verbatim}
##   pb bio1 bio12 bio16 bio17 bio5 bio6 bio7 bio8 biome
## 1  1  245  1042   503    77  336  138  199  276     1
## 2  1  202  2500   811   363  261  149  112  198     1
## 3  1  208  1471   654   162  289  121  168  225     1
## 4  1  255  3416  1491   194  330  190  139  248     1
## 5  1  131   846   452    47  229   21  208  135     1
## 6  1  233   679   286    77  356   84  272  276    13
\end{verbatim}

Ahora evaluamos las variables que están correlacionadas, es necesario
que dentro del modelo evitemos usar variables que muestren colinearidad
espacial. Para ello realizamos correlaciones entre variables para
determinar que variables están asociadas y poder decidir que variables
usar en el modelo.

\begin{Shaded}
\begin{Highlighting}[]
\KeywordTok{pairs}\NormalTok{(sdmdata[,}\DecValTok{2}\NormalTok{:}\DecValTok{5}\NormalTok{], }\DataTypeTok{cex=}\FloatTok{0.1}\NormalTok{)}
\end{Highlighting}
\end{Shaded}

\includegraphics{index_files/figure-latex/unnamed-chunk-19-1.pdf}

Como podemos observar las variables bio12 y bio16 están fuertemente
correlacionadas. Por ahora nos quedaremos con las variables que hemos
logrado obtener. Por ahora grabamos los datos para que estén disponibles
para los siguientes análisis.

\begin{Shaded}
\begin{Highlighting}[]
\KeywordTok{saveRDS}\NormalTok{(sdmdata, }\StringTok{"sdm.Rds"}\NormalTok{)}
\KeywordTok{saveRDS}\NormalTok{(presvals, }\StringTok{"pvals.Rds"}\NormalTok{)}
\end{Highlighting}
\end{Shaded}

\subsection{Ajuste del modelo, predicción y
evaluación}\label{ajuste-del-modelo-prediccion-y-evaluacion}

\subsubsection{Ajuste de un modelo}\label{ajuste-de-un-modelo}

Ajustar un SDM (modelo espacial) es técnicamente similar a otras
técnicas de modelado. Lo que haremos es identificar las variables
dependientes en independientes. De esta forma la variable dependiente
presencias-ausencias en relación a las variables ambientales. Vamos a
ajustar un modelo lineal generalizado con los datos que hemos levantado.

\begin{Shaded}
\begin{Highlighting}[]
\CommentTok{#leemos los datos}
\KeywordTok{library}\NormalTok{(dismo)}
\NormalTok{sdmdata <-}\StringTok{ }\KeywordTok{readRDS}\NormalTok{(}\StringTok{"sdm.Rds"}\NormalTok{)}
\NormalTok{presvals <-}\StringTok{ }\KeywordTok{readRDS}\NormalTok{(}\StringTok{"pvals.Rds"}\NormalTok{)}

\CommentTok{#Usamos unas pocas variables}
\NormalTok{m1 <-}\StringTok{ }\KeywordTok{glm}\NormalTok{(pb ~}\StringTok{ }\NormalTok{bio1 +}\StringTok{ }\NormalTok{bio5 +}\StringTok{ }\NormalTok{bio12, }\DataTypeTok{data=}\NormalTok{sdmdata)}
\KeywordTok{summary}\NormalTok{(m1)}
\end{Highlighting}
\end{Shaded}

\begin{verbatim}
## 
## Call:
## glm(formula = pb ~ bio1 + bio5 + bio12, data = sdmdata)
## 
## Deviance Residuals: 
##     Min       1Q   Median       3Q      Max  
## -0.5542  -0.2808  -0.2012   0.4465   0.9337  
## 
## Coefficients:
##               Estimate Std. Error t value Pr(>|t|)    
## (Intercept)  8.900e-01  1.002e-01   8.884  < 2e-16 ***
## bio1         2.141e-03  5.023e-04   4.263 2.31e-05 ***
## bio5        -3.394e-03  5.066e-04  -6.700 4.44e-11 ***
## bio12       -3.420e-05  2.431e-05  -1.407     0.16    
## ---
## Signif. codes:  0 '***' 0.001 '**' 0.01 '*' 0.05 '.' 0.1 ' ' 1
## 
## (Dispersion parameter for gaussian family taken to be 0.1774249)
## 
##     Null deviance: 126.87  on 669  degrees of freedom
## Residual deviance: 118.16  on 666  degrees of freedom
## AIC: 748.8
## 
## Number of Fisher Scoring iterations: 2
\end{verbatim}

Ahora tenemos nuestro modelo, pero como habíamos dicho, no nos interesa
saber si estas variables afectan significativamente la presencia de
especies, sino poder realizar predicciones a partir de este modelo.

\subsubsection{Predicciones}\label{predicciones}

Usaremos el modelo para predecir la probabilidad de ocurrencia de esta
especie en unas zonas con diferentes condiciones ambientales. Usaremos
la función \emph{predict} para proyectar la probabilidad de ocurrencia.

\begin{Shaded}
\begin{Highlighting}[]
\CommentTok{#generamos unos valores para cada una de las}
\CommentTok{#variables usadas en el modelo}

\NormalTok{bio1 =}\StringTok{ }\KeywordTok{c}\NormalTok{(}\DecValTok{40}\NormalTok{, }\DecValTok{150}\NormalTok{, }\DecValTok{200}\NormalTok{)}
\NormalTok{bio5 =}\StringTok{ }\KeywordTok{c}\NormalTok{(}\DecValTok{60}\NormalTok{, }\DecValTok{115}\NormalTok{, }\DecValTok{290}\NormalTok{)}
\NormalTok{bio12 =}\StringTok{ }\KeywordTok{c}\NormalTok{(}\DecValTok{600}\NormalTok{, }\DecValTok{1600}\NormalTok{, }\DecValTok{1700}\NormalTok{)}

\CommentTok{#Las unimos en un data.frame}
\NormalTok{pd =}\StringTok{ }\KeywordTok{data.frame}\NormalTok{(}\KeywordTok{cbind}\NormalTok{(bio1, bio5, bio12))}
\NormalTok{pd}
\end{Highlighting}
\end{Shaded}

\begin{verbatim}
##   bio1 bio5 bio12
## 1   40   60   600
## 2  150  115  1600
## 3  200  290  1700
\end{verbatim}

\begin{Shaded}
\begin{Highlighting}[]
\CommentTok{#predecimos la probabilidad de ocurrencia en esos sitios}
\KeywordTok{predict}\NormalTok{(m1, pd)}
\end{Highlighting}
\end{Shaded}

\begin{verbatim}
##         1         2         3 
## 0.7514777 0.7661183 0.2757831
\end{verbatim}

Como vemos la probabilidad de ocurrencia en los sitios 1 y 2 son mayores
al sitio 3. Pero, ¿Cómo las variables influencian la ocurrencia? Podemos
utilizar la función \emph{response} para ver cómo cada una de estas
variables determina la ocurrencia.

\begin{Shaded}
\begin{Highlighting}[]
\KeywordTok{response}\NormalTok{(m1)}
\end{Highlighting}
\end{Shaded}

\includegraphics{index_files/figure-latex/unnamed-chunk-23-1.pdf}

Como vemos bio5 tiene un efecto más fuerte sobre la probabilidad de
ocurrencia de la especie.

Ahora, como lo que nos interesa que las probabilidades de ocurrencia se
muestren en el espacio, utilizaremos el paquete \textbf{raster} y la
función predict para espacializar el modelo.

\begin{Shaded}
\begin{Highlighting}[]
\CommentTok{# Extraemos las variables}
\NormalTok{predictors <-}\StringTok{ }\KeywordTok{stack}\NormalTok{(}\KeywordTok{list.files}\NormalTok{(}\KeywordTok{file.path}\NormalTok{(}\KeywordTok{system.file}\NormalTok{(}\DataTypeTok{package=}\StringTok{"dismo"}\NormalTok{), }\StringTok{'ex'}\NormalTok{), }\DataTypeTok{pattern=}\StringTok{'grd$'}\NormalTok{, }\DataTypeTok{full.names=}\OtherTok{TRUE} \NormalTok{))}
\KeywordTok{names}\NormalTok{(predictors)}
\end{Highlighting}
\end{Shaded}

\begin{verbatim}
## [1] "bio1"  "bio12" "bio16" "bio17" "bio5"  "bio6"  "bio7"  "bio8"  "biome"
\end{verbatim}

\begin{Shaded}
\begin{Highlighting}[]
\CommentTok{#realizamos la predicción en base al modelo 1}
\NormalTok{p <-}\StringTok{ }\KeywordTok{predict}\NormalTok{(predictors, m1)}
\CommentTok{#graficamos}
\KeywordTok{plot}\NormalTok{(p)}
\end{Highlighting}
\end{Shaded}

\includegraphics{index_files/figure-latex/unnamed-chunk-24-1.pdf}

\subsubsection{Evaluación del modelo}\label{evaluacion-del-modelo}

Como vemos resulta ser bastante sencillo, hacer un modelo y mostrarlo en
el espacio. Sin embargo, lo que nos interesa es saber si este modelo
predice efectivamente la ocurrencia de las especies. Existen muchas
formas de evaluar el ajuste de un modelo y estas son dependientes del
modelo que se esté usando. Uno de los métodos más utilizados para
realizar una evaluación de los SDM es la validación cruzada
(cross-validation).

La validación cruzada consiste en crear un modelo con un conjunto de
datos de ``entrenamiento'' y probarlo con otro conjunto de datos de
ocurrencias conocidas, datos de ``prueba''. Por lo general, los datos de
entrenamiento y prueba se crean muestreando aleatoriamente (sin
reemplazo) el conjunto de datos que hemos logrado recabar.

Se pueden usar diferentes medidas para evaluar la calidad de una
predicción (Fielding y Bell, 1997, Liu et al., 2011; y Potts, para datos
de presencia y Elith 2006 para datos de abundancia), dependiendo del
objetivo del estudio. Muchas medidas para evaluar modelos basadas en
datos de presencia-ausencia son ``dependientes del umbral''. Eso
significa que primero se debe establecer un umbral sobre el cual se
considera que la especie ocurre (por ejemplo, 0.5, aunque 0.5 rara vez
es una opción sensata, por ejemplo, ver Lui et al. 2005). Los valores
pronosticados por encima de ese umbral indican una predicción de
``presencia'', y los valores por debajo del umbral indican ``ausencia''.

Las estadísticas más utilizadas, que son independientes del umbral son
el coeficiente de correlación y el área bajo la curva del operador
receptor (AUROC, generalmente abreviado como AUC). AUC es una medida de
correlación de rango. En datos imparciales, un AUC alto indica que los
sitios con valores de aptitud predichos altos tienden a ser áreas de
presencia conocida y las ubicaciones con valores de predicción modelo
más bajos tienden a ser áreas donde no se sabe que la especie esté
presente (ausente o un punto aleatorio). Un puntaje de AUC de 0.5
significa que el modelo es tan bueno como una suposición aleatoria. Ver
Phillips et al. (2006) para una discusión sobre el uso de AUC en el
contexto de datos de solo presencia en lugar de datos de presencia /
ausencia. Normalmente, modelos por arriba de 0.75 de AUC son modelos
considerados aceptables.

Para ejemplificar, vamos a realizar un modelo usando Bioclim un modelo
que usa solo datos de presencia.

\begin{Shaded}
\begin{Highlighting}[]
\CommentTok{#sacamos el 75% de los datos para entrenamiento}
\NormalTok{samp <-}\StringTok{ }\KeywordTok{sample}\NormalTok{(}\KeywordTok{nrow}\NormalTok{(sdmdata), }\KeywordTok{round}\NormalTok{(}\FloatTok{0.75} \NormalTok{*}\StringTok{ }\KeywordTok{nrow}\NormalTok{(sdmdata)))}
\NormalTok{traindata <-}\StringTok{ }\NormalTok{sdmdata[samp,]}

\CommentTok{#Eliminamos las ausencias}
\NormalTok{traindata <-}\StringTok{ }\NormalTok{traindata[traindata[,}\DecValTok{1}\NormalTok{] ==}\StringTok{ }\DecValTok{1}\NormalTok{, }\DecValTok{2}\NormalTok{:}\DecValTok{9}\NormalTok{]}

\CommentTok{#El 25% restante queda como datos de prueba}
\NormalTok{testdata <-}\StringTok{ }\NormalTok{sdmdata[-samp,]}

\CommentTok{#Hacemos el modelo con datos de entrenamiento}
\NormalTok{bc <-}\StringTok{ }\KeywordTok{bioclim}\NormalTok{(traindata)}

\CommentTok{#evaluamos presencias y ausencias}
\NormalTok{e <-}\StringTok{ }\KeywordTok{evaluate}\NormalTok{(testdata[testdata==}\DecValTok{1}\NormalTok{,], testdata[testdata==}\DecValTok{0}\NormalTok{,], bc)}
\NormalTok{e}
\end{Highlighting}
\end{Shaded}

\begin{verbatim}
## class          : ModelEvaluation 
## n presences    : 41 
## n absences     : 127 
## AUC            : 0.6497983 
## cor            : 0.1867107 
## max TPR+TNR at : 0.04641163
\end{verbatim}

\begin{Shaded}
\begin{Highlighting}[]
\CommentTok{#Graficamos el resultado}
\KeywordTok{plot}\NormalTok{(e, }\StringTok{'ROC'}\NormalTok{)}
\end{Highlighting}
\end{Shaded}

\includegraphics{index_files/figure-latex/unnamed-chunk-25-1.pdf}

Lo que podemos ver es que nuestro modelo no es diferente a una
distribución aleatoria, en otras palabras aunque tenemos un modelo este
no predice la distribución de esta especie, la predicción es cercana a
un modelo de ocurrencia aleatoria. Más adelante volveremos a la
evaluación de los modelos.

\subsection{Métodos de modelamiento}\label{metodos-de-modelamiento}

\subsubsection{Tipos de algoritmos
usados}\label{tipos-de-algoritmos-usados}

Una gran cantidad de algoritmos han sido utilizados para desarrollar
SDM. Estos algoritmos pueden ser clasificados en tres tipos; i)
envuelta, ii) regresión y iii) aprendizaje automático (Machine
learning). Los algoritmos de envuelta consideran solamente los datos de
presencia. Los otros dos tipos de algoritmos usan datos de presencia y
ausencia.

Vamos a utilizar datos de una especie del \emph{Desmodus rotundus} que
hemos limpiado. Además, extraeremos los datos de las variables
independientes que usaremos y, finalmente, dividiremos los datos en
entrenamiento y validación.

\begin{Shaded}
\begin{Highlighting}[]
\KeywordTok{library}\NormalTok{(dismo)}
\KeywordTok{library}\NormalTok{(maptools)}
\KeywordTok{data}\NormalTok{(wrld_simpl)}

\CommentTok{#extraemos los datos bioclimáticos}
\NormalTok{predictors <-}\StringTok{ }\KeywordTok{stack}\NormalTok{(}\KeywordTok{list.files}\NormalTok{(}\KeywordTok{file.path}\NormalTok{(}\KeywordTok{system.file}\NormalTok{(}\DataTypeTok{package=}\StringTok{"dismo"}\NormalTok{), }\StringTok{'ex'}\NormalTok{), }\DataTypeTok{pattern=}\StringTok{'grd$'}\NormalTok{, }\DataTypeTok{full.names=}\OtherTok{TRUE} \NormalTok{))}

\CommentTok{#obtenemos los datos de cada predictor para nuestros datos de presencia}
\NormalTok{presvals <-}\StringTok{ }\KeywordTok{extract}\NormalTok{(predictors, desFD)}
\KeywordTok{set.seed}\NormalTok{(}\DecValTok{0}\NormalTok{)}

\CommentTok{#Generamos datos de fondo}
\NormalTok{backgr <-}\StringTok{ }\KeywordTok{randomPoints}\NormalTok{(predictors, }\DecValTok{500}\NormalTok{)}

\CommentTok{#Extraemos los datos de cada predictor de los datos de fondo}
\NormalTok{absvals <-}\StringTok{ }\KeywordTok{extract}\NormalTok{(predictors, backgr)}

\CommentTok{#Generamos un vector de presencia ausencia}
\NormalTok{pb <-}\StringTok{ }\KeywordTok{c}\NormalTok{(}\KeywordTok{rep}\NormalTok{(}\DecValTok{1}\NormalTok{, }\KeywordTok{nrow}\NormalTok{(presvals)), }\KeywordTok{rep}\NormalTok{(}\DecValTok{0}\NormalTok{, }\KeywordTok{nrow}\NormalTok{(absvals)))}

\CommentTok{#Unimos todo}
\NormalTok{sdmdata <-}\StringTok{ }\KeywordTok{data.frame}\NormalTok{(}\KeywordTok{cbind}\NormalTok{(pb, }\KeywordTok{rbind}\NormalTok{(presvals, absvals)))}
\NormalTok{sdmdata[,}\StringTok{'biome'}\NormalTok{] <-}\StringTok{ }\KeywordTok{as.factor}\NormalTok{(sdmdata[,}\StringTok{'biome'}\NormalTok{])}
\end{Highlighting}
\end{Shaded}

Debido a que algunos algoritmos no admiten variables categóricas
eliminaremos del stack la variable de bioma.

\begin{Shaded}
\begin{Highlighting}[]
\NormalTok{pred_nf <-}\StringTok{ }\KeywordTok{dropLayer}\NormalTok{(predictors, }\StringTok{'biome'}\NormalTok{)}
\end{Highlighting}
\end{Shaded}

Ahora dividimos los datos en entrenamiento y prueba.

\begin{Shaded}
\begin{Highlighting}[]
\KeywordTok{set.seed}\NormalTok{(}\DecValTok{0}\NormalTok{)}

\CommentTok{#generamos 5 grupos aleatorios}
\NormalTok{group <-}\StringTok{ }\KeywordTok{kfold}\NormalTok{(desmodFD, }\DecValTok{5}\NormalTok{)}

\CommentTok{#Escogemos un grupo de prueba y el resto de entrenamiento}
\NormalTok{pres_train <-}\StringTok{ }\NormalTok{desmodFD[group !=}\StringTok{ }\DecValTok{1}\NormalTok{, ]}
\NormalTok{pres_test <-}\StringTok{ }\NormalTok{desmodFD[group ==}\StringTok{ }\DecValTok{1}\NormalTok{, ]}

\CommentTok{#Restringimos la predicción }
\NormalTok{ext <-}\StringTok{ }\KeywordTok{extent}\NormalTok{(-}\DecValTok{110}\NormalTok{, -}\DecValTok{32}\NormalTok{, -}\DecValTok{33}\NormalTok{, }\DecValTok{30}\NormalTok{)}
\end{Highlighting}
\end{Shaded}

Generamos los datos de fondo que deberán ser separados entre
entrenamiento y prueba. Para evitar que los datos de entrenamiento
tengan NA o estén fuera del marco establecido vamos a usar el argumento
\emph{extf} que nos permite limitar la generación de los puntos de fondo
hasta un 12.5\% del limite.

\begin{Shaded}
\begin{Highlighting}[]
\CommentTok{#Generamos los datos de fondo}
\KeywordTok{set.seed}\NormalTok{(}\DecValTok{10}\NormalTok{)}
\NormalTok{backg <-}\StringTok{ }\KeywordTok{randomPoints}\NormalTok{(pred_nf, }\DataTypeTok{n=}\DecValTok{1000}\NormalTok{, }\DataTypeTok{ext=}\NormalTok{ext, }\DataTypeTok{extf =} \NormalTok{-}\FloatTok{1.25}\NormalTok{)}
\KeywordTok{colnames}\NormalTok{(backg) <-}\StringTok{ }\KeywordTok{c}\NormalTok{(}\StringTok{'lon'}\NormalTok{, }\StringTok{'lat'}\NormalTok{)}
\NormalTok{group <-}\StringTok{ }\KeywordTok{kfold}\NormalTok{(backg, }\DecValTok{5}\NormalTok{)}

\CommentTok{#separamos datos de entrenamiento y prueba}
\NormalTok{backg_train <-}\StringTok{ }\NormalTok{backg[group !=}\StringTok{ }\DecValTok{1}\NormalTok{, ]}
\NormalTok{backg_test <-}\StringTok{ }\NormalTok{backg[group ==}\StringTok{ }\DecValTok{1}\NormalTok{, ]}
\end{Highlighting}
\end{Shaded}

Ahora graficamos la información que hemos obtenido para confirmar que
todo esté bien. Usaremos uno de los predictores como marco de la figura.

\begin{Shaded}
\begin{Highlighting}[]
\CommentTok{#Extraemos los datos del raster }
\NormalTok{r <-}\StringTok{ }\KeywordTok{raster}\NormalTok{(pred_nf, }\DecValTok{1}\NormalTok{)}

\CommentTok{#Graficamos el raster}
\KeywordTok{plot}\NormalTok{(!}\KeywordTok{is.na}\NormalTok{(r), }\DataTypeTok{col=}\KeywordTok{c}\NormalTok{(}\StringTok{'white'}\NormalTok{, }\StringTok{'light grey'}\NormalTok{), }\DataTypeTok{legend=}\OtherTok{FALSE}\NormalTok{)}

\CommentTok{#graficamos la ventana}
\KeywordTok{plot}\NormalTok{(ext, }\DataTypeTok{add=}\OtherTok{TRUE}\NormalTok{, }\DataTypeTok{col=}\StringTok{'red'}\NormalTok{, }\DataTypeTok{lwd=}\DecValTok{2}\NormalTok{)}
\CommentTok{#graficamos los puntos de fondo}
\KeywordTok{points}\NormalTok{(backg_train, }\DataTypeTok{pch=}\StringTok{'-'}\NormalTok{, }\DataTypeTok{cex=}\FloatTok{0.9}\NormalTok{, }\DataTypeTok{col=}\StringTok{'yellow'}\NormalTok{)}
\KeywordTok{points}\NormalTok{(backg_test, }\DataTypeTok{pch=}\StringTok{'-'}\NormalTok{,  }\DataTypeTok{cex=}\FloatTok{0.9}\NormalTok{, }\DataTypeTok{col=}\StringTok{'black'}\NormalTok{)}
\CommentTok{#graficamos los puntos de ocurrencia}
\KeywordTok{points}\NormalTok{(pres_train$lon, pres_train$lat, }\DataTypeTok{pch=} \StringTok{'+'}\NormalTok{, }\DataTypeTok{col=}\StringTok{'green'}\NormalTok{)}
\KeywordTok{points}\NormalTok{(pres_test$lon, pres_test$lat, }\DataTypeTok{pch=}\StringTok{'+'}\NormalTok{, }\DataTypeTok{col=}\StringTok{'blue'}\NormalTok{)}
\end{Highlighting}
\end{Shaded}

\includegraphics{index_files/figure-latex/unnamed-chunk-30-1.pdf}

Al parecer todo está donde debería estar, ahora podemos ajustar los
diferentes modelos utilizando la información que hemos preparado.

\subsubsection{Algoritmos de Envuelta}\label{algoritmos-de-envuelta}

A continuación vamos a usar tres diferentes algoritmos para realizar
nuestros modelos; BIOCLIM, DOMAIN y distancia de MAHALANOBIS. Como lo
mencionamos anteriormente estos algoritmos usan únicamente los datos de
presencia. Aunque estos modelos han sido ampliamente utilizados, según
algunos autores, estos modelos son menos efectivos que otros,
particularmente cuando son utilizados para evaluar cambio climático. El
algoritmo BIOCLIM calcula la similitud de una ubicación comparando los
valores de las variables ambientales en cualquier ubicación con una
distribución porcentual de los valores en ubicaciones conocidas de
ocurrencia (``sitios de entrenamiento'').

\begin{Shaded}
\begin{Highlighting}[]
\CommentTok{#Ajustamos el algoritmo a nuestros datos}
\NormalTok{bc <-}\StringTok{ }\KeywordTok{bioclim}\NormalTok{(pred_nf, pres_train[,}\KeywordTok{c}\NormalTok{(}\StringTok{"lon"}\NormalTok{,}\StringTok{"lat"}\NormalTok{)])}
\CommentTok{#evaluamos el modelo}
\NormalTok{e <-}\StringTok{ }\KeywordTok{evaluate}\NormalTok{(pres_test[,}\KeywordTok{c}\NormalTok{(}\StringTok{"lon"}\NormalTok{,}\StringTok{"lat"}\NormalTok{)], backg_test[,}\KeywordTok{c}\NormalTok{(}\StringTok{"lon"}\NormalTok{,}\StringTok{"lat"}\NormalTok{)], bc, pred_nf)}
\NormalTok{e}
\end{Highlighting}
\end{Shaded}

\begin{verbatim}
## class          : ModelEvaluation 
## n presences    : 34 
## n absences     : 200 
## AUC            : 0.6017647 
## cor            : 0.04966259 
## max TPR+TNR at : 0.05872353
\end{verbatim}

Como vemos el modelo generado con BIOCLIM no ajusta adecuadamente (AUC
0.61).

\begin{Shaded}
\begin{Highlighting}[]
\CommentTok{#Definimos a que nivel se realizará el corte de presencia-ausencia }
\NormalTok{tr <-}\StringTok{ }\KeywordTok{threshold}\NormalTok{(e, }\StringTok{'spec_sens'}\NormalTok{)}
\NormalTok{tr}
\end{Highlighting}
\end{Shaded}

\begin{verbatim}
## [1] 0.05872353
\end{verbatim}

\begin{Shaded}
\begin{Highlighting}[]
\CommentTok{#Extrapolamos los datos a un área de interés}
\NormalTok{pb <-}\StringTok{ }\KeywordTok{predict}\NormalTok{(pred_nf, bc, }\DataTypeTok{ext=}\NormalTok{ext, }\DataTypeTok{progress=}\StringTok{''}\NormalTok{)}

\CommentTok{#graficamos el modelo}
\KeywordTok{par}\NormalTok{(}\DataTypeTok{mfrow=}\KeywordTok{c}\NormalTok{(}\DecValTok{1}\NormalTok{,}\DecValTok{2}\NormalTok{))}
\KeywordTok{plot}\NormalTok{(pb, }\DataTypeTok{main=}\StringTok{'Bioclim, raw values'}\NormalTok{)}
\KeywordTok{plot}\NormalTok{(wrld_simpl, }\DataTypeTok{add=}\OtherTok{TRUE}\NormalTok{, }\DataTypeTok{border=}\StringTok{'dark grey'}\NormalTok{)}
\KeywordTok{plot}\NormalTok{(pb >}\StringTok{ }\NormalTok{tr, }\DataTypeTok{main=}\StringTok{'presence/absence'}\NormalTok{)}
\KeywordTok{plot}\NormalTok{(wrld_simpl, }\DataTypeTok{add=}\OtherTok{TRUE}\NormalTok{, }\DataTypeTok{border=}\StringTok{'dark grey'}\NormalTok{)}
\KeywordTok{points}\NormalTok{(pres_train, }\DataTypeTok{pch=}\StringTok{'+'}\NormalTok{) }
\end{Highlighting}
\end{Shaded}

\includegraphics{index_files/figure-latex/unnamed-chunk-32-1.pdf}

El algoritmo de Domain (Carpenter et al. 1993) calcula la distancia de
Gower entre las variables ambientales en cualquier ubicación y de estas
en las ubicaciones de ocurrencia (``sitios de entrenamiento''). A
continuación, ajustamos un modelo de dominio, lo evaluamos y hacemos una
predicción. Mapeamos la predicción, así como un mapa clasificado
subjetivamente en presencia / ausencia.

\begin{Shaded}
\begin{Highlighting}[]
\CommentTok{#generamos el modelo}
\NormalTok{dm <-}\StringTok{ }\KeywordTok{domain}\NormalTok{(pred_nf, pres_train[,}\KeywordTok{c}\NormalTok{(}\StringTok{"lon"}\NormalTok{,}\StringTok{"lat"}\NormalTok{)])}

\CommentTok{#lo evaluamos}
\NormalTok{e <-}\StringTok{ }\KeywordTok{evaluate}\NormalTok{(pres_test[,}\KeywordTok{c}\NormalTok{(}\StringTok{"lon"}\NormalTok{,}\StringTok{"lat"}\NormalTok{)], backg_test, dm, pred_nf)}
\NormalTok{e}
\end{Highlighting}
\end{Shaded}

\begin{verbatim}
## class          : ModelEvaluation 
## n presences    : 34 
## n absences     : 200 
## AUC            : 0.5425 
## cor            : 0.1253585 
## max TPR+TNR at : 0.4919958
\end{verbatim}

\begin{Shaded}
\begin{Highlighting}[]
\CommentTok{#predecimos en el espacio}
\NormalTok{pd =}\StringTok{ }\KeywordTok{predict}\NormalTok{(pred_nf, dm, }\DataTypeTok{ext=}\NormalTok{ext, }\DataTypeTok{progress=}\StringTok{''}\NormalTok{)}

\CommentTok{#graficamos el modelo}
\KeywordTok{par}\NormalTok{(}\DataTypeTok{mfrow=}\KeywordTok{c}\NormalTok{(}\DecValTok{1}\NormalTok{,}\DecValTok{2}\NormalTok{))}
\KeywordTok{plot}\NormalTok{(pd, }\DataTypeTok{main=}\StringTok{'Domain, raw values'}\NormalTok{)}
\KeywordTok{plot}\NormalTok{(wrld_simpl, }\DataTypeTok{add=}\OtherTok{TRUE}\NormalTok{, }\DataTypeTok{border=}\StringTok{'dark grey'}\NormalTok{)}
\NormalTok{tr <-}\StringTok{ }\KeywordTok{threshold}\NormalTok{(e, }\StringTok{'spec_sens'}\NormalTok{)}
\KeywordTok{plot}\NormalTok{(pd >}\StringTok{ }\NormalTok{tr, }\DataTypeTok{main=}\StringTok{'presence/absence'}\NormalTok{)}
\KeywordTok{plot}\NormalTok{(wrld_simpl, }\DataTypeTok{add=}\OtherTok{TRUE}\NormalTok{, }\DataTypeTok{border=}\StringTok{'dark grey'}\NormalTok{)}
\KeywordTok{points}\NormalTok{(pres_train, }\DataTypeTok{pch=}\StringTok{'+'}\NormalTok{)}
\end{Highlighting}
\end{Shaded}

\includegraphics{index_files/figure-latex/unnamed-chunk-33-1.pdf}

Finalmente ajustamos el algoritmo Malahanobis. La función \emph{mahal}
implementa un modelo de distribución de especies basado en la distancia
de Mahalanobis (Mahalanobis, 1936). La distancia de Mahalanobis tiene en
cuenta las correlaciones de las variables en el conjunto de datos, y no
depende de la escala de mediciones.

\begin{Shaded}
\begin{Highlighting}[]
\CommentTok{#Ajustamos el modelo}
\NormalTok{mm <-}\StringTok{ }\KeywordTok{mahal}\NormalTok{(pred_nf, pres_train[,}\KeywordTok{c}\NormalTok{(}\StringTok{"lon"}\NormalTok{,}\StringTok{"lat"}\NormalTok{)])}

\CommentTok{#lo evaluamos}
\NormalTok{em <-}\StringTok{ }\KeywordTok{evaluate}\NormalTok{(pres_test[,}\KeywordTok{c}\NormalTok{(}\StringTok{"lon"}\NormalTok{,}\StringTok{"lat"}\NormalTok{)], backg_test, mm, pred_nf)}
\NormalTok{em}
\end{Highlighting}
\end{Shaded}

\begin{verbatim}
## class          : ModelEvaluation 
## n presences    : 34 
## n absences     : 200 
## AUC            : 0.7479412 
## cor            : 0.1532754 
## max TPR+TNR at : -1.133013
\end{verbatim}

\begin{Shaded}
\begin{Highlighting}[]
\CommentTok{#predecimos}
\NormalTok{pm =}\StringTok{ }\KeywordTok{predict}\NormalTok{(pred_nf, mm, }\DataTypeTok{ext=}\NormalTok{ext, }\DataTypeTok{progress=}\StringTok{''}\NormalTok{)}

\CommentTok{#Los graficamos}
\KeywordTok{par}\NormalTok{(}\DataTypeTok{mfrow=}\KeywordTok{c}\NormalTok{(}\DecValTok{1}\NormalTok{,}\DecValTok{2}\NormalTok{))}
\NormalTok{pm[pm <}\StringTok{ }\NormalTok{-}\DecValTok{10}\NormalTok{] <-}\StringTok{ }\NormalTok{-}\DecValTok{10}
\KeywordTok{plot}\NormalTok{(pm, }\DataTypeTok{main=}\StringTok{'Mahalanobis distance'}\NormalTok{)}
\KeywordTok{plot}\NormalTok{(wrld_simpl, }\DataTypeTok{add=}\OtherTok{TRUE}\NormalTok{, }\DataTypeTok{border=}\StringTok{'dark grey'}\NormalTok{)}
\NormalTok{tr <-}\StringTok{ }\KeywordTok{threshold}\NormalTok{(e, }\StringTok{'spec_sens'}\NormalTok{)}
\KeywordTok{plot}\NormalTok{(pm >}\StringTok{ }\NormalTok{tr, }\DataTypeTok{main=}\StringTok{'presence/absence'}\NormalTok{)}
\KeywordTok{plot}\NormalTok{(wrld_simpl, }\DataTypeTok{add=}\OtherTok{TRUE}\NormalTok{, }\DataTypeTok{border=}\StringTok{'dark grey'}\NormalTok{)}
\KeywordTok{points}\NormalTok{(pres_train, }\DataTypeTok{pch=}\StringTok{'+'}\NormalTok{)}
\end{Highlighting}
\end{Shaded}

\includegraphics{index_files/figure-latex/unnamed-chunk-34-1.pdf}

\subsubsection{Modelos de regresiones}\label{modelos-de-regresiones}

Los modelos de regresiones se ajustan a los datos de presencia y
ausencia (fondo). No podemos ajustar el modelo con un RasterStack y
puntos. Lo que haremos es extraer los valores de los datos ambientales y
ajustar los modelos con estos valores.

\begin{Shaded}
\begin{Highlighting}[]
\CommentTok{#pegamos los datos de presencia y ausencia}
\CommentTok{#Las coordenadas}
\NormalTok{train <-}\StringTok{ }\KeywordTok{rbind}\NormalTok{(pres_train[,}\KeywordTok{c}\NormalTok{(}\StringTok{"lon"}\NormalTok{,}\StringTok{"lat"}\NormalTok{)], backg_train[,}\KeywordTok{c}\NormalTok{(}\StringTok{"lon"}\NormalTok{,}\StringTok{"lat"}\NormalTok{)])}
\CommentTok{#la variable de ocurrencia}
\NormalTok{pb_train <-}\StringTok{ }\KeywordTok{c}\NormalTok{(}\KeywordTok{rep}\NormalTok{(}\DecValTok{1}\NormalTok{, }\KeywordTok{nrow}\NormalTok{(pres_train)), }\KeywordTok{rep}\NormalTok{(}\DecValTok{0}\NormalTok{, }\KeywordTok{nrow}\NormalTok{(backg_train)))}

\CommentTok{#Extraemos los datos ambientales de entrenamiento}
\NormalTok{envtrain <-}\StringTok{ }\KeywordTok{extract}\NormalTok{(predictors, train)}
\NormalTok{envtrain <-}\StringTok{ }\KeywordTok{data.frame}\NormalTok{( }\KeywordTok{cbind}\NormalTok{(}\DataTypeTok{pa=}\NormalTok{pb_train, envtrain) )}
\NormalTok{envtrain[,}\StringTok{'biome'}\NormalTok{] =}\StringTok{ }\KeywordTok{factor}\NormalTok{(envtrain[,}\StringTok{'biome'}\NormalTok{], }\DataTypeTok{levels=}\DecValTok{1}\NormalTok{:}\DecValTok{14}\NormalTok{)}
\KeywordTok{head}\NormalTok{(envtrain)}
\end{Highlighting}
\end{Shaded}

\begin{verbatim}
##   pa bio1 bio12 bio16 bio17 bio5 bio6 bio7 bio8 biome
## 1  1  245  1042   503    77  336  138  199  276     1
## 2  1  208  1471   654   162  289  121  168  225     1
## 3  1  255  3416  1491   194  330  190  139  248     1
## 4  1  131   846   452    47  229   21  208  135     1
## 5  1  270  1190   501   141  322  216  106  268    13
## 6  1  210  1429   638    81  302  127  175  218     3
\end{verbatim}

\begin{Shaded}
\begin{Highlighting}[]
\CommentTok{#Extraemos los datos ambientales de evaluación}
\NormalTok{testpres <-}\StringTok{ }\KeywordTok{data.frame}\NormalTok{( }\KeywordTok{extract}\NormalTok{(predictors, pres_test[,}\KeywordTok{c}\NormalTok{(}\StringTok{"lon"}\NormalTok{,}\StringTok{"lat"}\NormalTok{)]) )}
\NormalTok{testbackg <-}\StringTok{ }\KeywordTok{data.frame}\NormalTok{( }\KeywordTok{extract}\NormalTok{(predictors, backg_test[,}\KeywordTok{c}\NormalTok{(}\StringTok{"lon"}\NormalTok{,}\StringTok{"lat"}\NormalTok{)]) )}
\NormalTok{testpres[ ,}\StringTok{'biome'}\NormalTok{] =}\StringTok{ }\KeywordTok{factor}\NormalTok{(testpres[ ,}\StringTok{'biome'}\NormalTok{], }\DataTypeTok{levels=}\DecValTok{1}\NormalTok{:}\DecValTok{14}\NormalTok{)}
\NormalTok{testbackg[ ,}\StringTok{'biome'}\NormalTok{] =}\StringTok{ }\KeywordTok{factor}\NormalTok{(testbackg[ ,}\StringTok{'biome'}\NormalTok{], }\DataTypeTok{levels=}\DecValTok{1}\NormalTok{:}\DecValTok{14}\NormalTok{)}
\end{Highlighting}
\end{Shaded}

\textbf{Modelos Lineales Generalizados} Un modelo lineal generalizado
(GLM) es una generalización de la regresión de mínimos cuadrados
ordinarios. Los modelos se ajustan utilizando la máxima verosimilitud y
permitiendo que el modelo lineal se relacione con la variable de
respuesta a través de una función de enlace y permitiendo que la
magnitud de la varianza de cada medición sea una función de su valor
predicho. Dependiendo de cómo se especifique un GLM, puede ser
equivalente a la regresión lineal (múltiple), la regresión logística o
la regresión de Poisson. Ver Guisan et al (2002) para una visión general
del uso de GLM en el modelado de distribución de especies.

A continuación ajustaremos dos modelos lineales generalizados con
diferente función de enlace.

\begin{Shaded}
\begin{Highlighting}[]
\CommentTok{#Usamos una regresión logística}
\NormalTok{gm1 <-}\StringTok{ }\KeywordTok{glm}\NormalTok{(pa ~}\StringTok{ }\NormalTok{bio16 +}\StringTok{ }\NormalTok{bio5 +}\StringTok{ }\NormalTok{bio6 +}\StringTok{ }\NormalTok{bio17 +}\StringTok{ }\NormalTok{bio12,}
            \DataTypeTok{family =} \KeywordTok{binomial}\NormalTok{(}\DataTypeTok{link =} \StringTok{"logit"}\NormalTok{), }\DataTypeTok{data=}\NormalTok{envtrain)}
\KeywordTok{summary}\NormalTok{(gm1)}
\end{Highlighting}
\end{Shaded}

\begin{verbatim}
## 
## Call:
## glm(formula = pa ~ bio16 + bio5 + bio6 + bio17 + bio12, family = binomial(link = "logit"), 
##     data = envtrain)
## 
## Deviance Residuals: 
##     Min       1Q   Median       3Q      Max  
## -1.6339  -0.5719  -0.4226  -0.2666   2.6045  
## 
## Coefficients:
##               Estimate Std. Error z value Pr(>|z|)    
## (Intercept)  2.9804205  0.6866848   4.340 1.42e-05 ***
## bio16        0.0050720  0.0014054   3.609 0.000307 ***
## bio5        -0.0190642  0.0026575  -7.174 7.30e-13 ***
## bio6         0.0036771  0.0019862   1.851 0.064118 .  
## bio17        0.0008156  0.0018642   0.437 0.661761    
## bio12       -0.0019073  0.0007152  -2.667 0.007655 ** 
## ---
## Signif. codes:  0 '***' 0.001 '**' 0.01 '*' 0.05 '.' 0.1 ' ' 1
## 
## (Dispersion parameter for binomial family taken to be 1)
## 
##     Null deviance: 775.88  on 935  degrees of freedom
## Residual deviance: 679.67  on 930  degrees of freedom
## AIC: 691.67
## 
## Number of Fisher Scoring iterations: 5
\end{verbatim}

\begin{Shaded}
\begin{Highlighting}[]
\CommentTok{#Ajustamos un GLM con una función gaussian}

\NormalTok{gm2 <-}\StringTok{ }\KeywordTok{glm}\NormalTok{(pa ~}\StringTok{  }\NormalTok{bio1+bio5 +}\StringTok{ }\NormalTok{bio6 +}\StringTok{ }\NormalTok{bio7 +}\StringTok{ }\NormalTok{bio8 +}\StringTok{ }\NormalTok{bio12 +}\StringTok{ }\NormalTok{bio16 +}\StringTok{ }\NormalTok{bio17,}
            \DataTypeTok{family =} \KeywordTok{gaussian}\NormalTok{(}\DataTypeTok{link =} \StringTok{"identity"}\NormalTok{), }\DataTypeTok{data=}\NormalTok{envtrain)}

\CommentTok{#Evaluamos el modelo 1}
\NormalTok{ge1 <-}\StringTok{ }\KeywordTok{evaluate}\NormalTok{(testpres, testbackg, gm1)}
\NormalTok{ge1}
\end{Highlighting}
\end{Shaded}

\begin{verbatim}
## class          : ModelEvaluation 
## n presences    : 34 
## n absences     : 200 
## AUC            : 0.7447059 
## cor            : 0.3147866 
## max TPR+TNR at : -1.825892
\end{verbatim}

\begin{Shaded}
\begin{Highlighting}[]
\CommentTok{#Evaluamos el modelo 2}
\NormalTok{ge2 <-}\StringTok{ }\KeywordTok{evaluate}\NormalTok{(testpres, testbackg, gm2)}
\NormalTok{ge2}
\end{Highlighting}
\end{Shaded}

\begin{verbatim}
## class          : ModelEvaluation 
## n presences    : 34 
## n absences     : 200 
## AUC            : 0.7266176 
## cor            : 0.313569 
## max TPR+TNR at : 0.1751886
\end{verbatim}

\begin{Shaded}
\begin{Highlighting}[]
\CommentTok{#Predecimos la distribución con el primer modelo}
\NormalTok{pg1 <-}\StringTok{ }\KeywordTok{predict}\NormalTok{(predictors, gm1, }\DataTypeTok{ext=}\NormalTok{ext)}
\NormalTok{pg2 <-}\StringTok{ }\KeywordTok{predict}\NormalTok{(predictors, gm2, }\DataTypeTok{ext=}\NormalTok{ext)}
\CommentTok{#Graficamos}
\KeywordTok{par}\NormalTok{(}\DataTypeTok{mfrow=}\KeywordTok{c}\NormalTok{(}\DecValTok{1}\NormalTok{,}\DecValTok{2}\NormalTok{))}
\KeywordTok{plot}\NormalTok{(pg1, }\DataTypeTok{main=}\StringTok{'GLM/logistic, raw values'}\NormalTok{)}
\KeywordTok{plot}\NormalTok{(wrld_simpl, }\DataTypeTok{add=}\OtherTok{TRUE}\NormalTok{, }\DataTypeTok{border=}\StringTok{'dark grey'}\NormalTok{)}
\NormalTok{tr <-}\StringTok{ }\KeywordTok{threshold}\NormalTok{(ge1, }\StringTok{'spec_sens'}\NormalTok{)}
\KeywordTok{plot}\NormalTok{(pg1 >}\StringTok{ }\NormalTok{tr, }\DataTypeTok{main=}\StringTok{'presence/absence'}\NormalTok{)}
\KeywordTok{plot}\NormalTok{(wrld_simpl, }\DataTypeTok{add=}\OtherTok{TRUE}\NormalTok{, }\DataTypeTok{border=}\StringTok{'dark grey'}\NormalTok{)}
\KeywordTok{points}\NormalTok{(pres_train, }\DataTypeTok{pch=}\StringTok{'+'}\NormalTok{)}
\KeywordTok{points}\NormalTok{(backg_train, }\DataTypeTok{pch=}\StringTok{'-'}\NormalTok{, }\DataTypeTok{cex=}\FloatTok{0.25}\NormalTok{)}
\end{Highlighting}
\end{Shaded}

\includegraphics{index_files/figure-latex/unnamed-chunk-36-1.pdf}

\subsubsection{Modelos de aprendizaje
automático}\label{modelos-de-aprendizaje-automatico}

Dentro de los algoritmos de aprendizaje automatizado MaxEnt (abreviatura
de ``Entropía máxima''; Phillips et al., 2006) es el algoritmo SDM más
utilizado. Maxent se encuentra implementado en Java pero puede ser
ejecutado desde el paquete \textbf{dismo}. Si se encuentra con una
versión de R superior a 3.5, Maxent no se encuentra disponible para esta
versión. Sin embargo, puede descargar Maxent haciendo clic
\href{http://biodiversityinformatics.amnh.org/open_source/maxent/}{aquí}.
Una vez descargada esta versión vaya a la libreria de R, busque el
paquete dismo y la carpeta java, copie y peque el ejecutable de maxent
en esta carpeta. La dirección debería ser similar a esta
``\ldots{}.R/win-library/3.5/dismo/java''

\begin{Shaded}
\begin{Highlighting}[]
\CommentTok{#Cargamos maxent}
\KeywordTok{maxent}\NormalTok{()}
\end{Highlighting}
\end{Shaded}

\begin{verbatim}
## Loading required namespace: rJava
\end{verbatim}

\begin{verbatim}
## This is MaxEnt version 3.4.1
\end{verbatim}

\begin{Shaded}
\begin{Highlighting}[]
\CommentTok{#Ajustamos el modelo}
\CommentTok{#Es necesario informar a maxent las variables que se encuentran como factores}
\NormalTok{mx <-}\StringTok{ }\KeywordTok{maxent}\NormalTok{(predictors, pres_train[,}\KeywordTok{c}\NormalTok{(}\StringTok{"lon"}\NormalTok{, }\StringTok{"lat"}\NormalTok{)], }\DataTypeTok{factors=}\StringTok{'biome'}\NormalTok{)}

\CommentTok{#Graficamos las variables y su importancia}
\KeywordTok{plot}\NormalTok{(mx)}
\end{Highlighting}
\end{Shaded}

\includegraphics{index_files/figure-latex/unnamed-chunk-37-1.pdf}

\begin{Shaded}
\begin{Highlighting}[]
\KeywordTok{response}\NormalTok{(mx)}
\end{Highlighting}
\end{Shaded}

\includegraphics{index_files/figure-latex/unnamed-chunk-37-2.pdf}

\begin{Shaded}
\begin{Highlighting}[]
\CommentTok{#Realizamos la evaluación del modelo}
\NormalTok{emx <-}\StringTok{ }\KeywordTok{evaluate}\NormalTok{(pres_test[,}\KeywordTok{c}\NormalTok{(}\StringTok{"lon"}\NormalTok{, }\StringTok{"lat"}\NormalTok{)], backg_test[,}\KeywordTok{c}\NormalTok{(}\StringTok{"lon"}\NormalTok{, }\StringTok{"lat"}\NormalTok{)], mx, predictors)}
\NormalTok{emx}
\end{Highlighting}
\end{Shaded}

\begin{verbatim}
## class          : ModelEvaluation 
## n presences    : 34 
## n absences     : 200 
## AUC            : 0.8197059 
## cor            : 0.5213534 
## max TPR+TNR at : 0.6323841
\end{verbatim}

\begin{Shaded}
\begin{Highlighting}[]
\CommentTok{#Predecimos}
\NormalTok{px <-}\StringTok{ }\KeywordTok{predict}\NormalTok{(predictors, mx, }\DataTypeTok{ext=}\NormalTok{ext, }\DataTypeTok{progress=}\StringTok{''}\NormalTok{)}

\CommentTok{#Graficamos}
\KeywordTok{par}\NormalTok{(}\DataTypeTok{mfrow=}\KeywordTok{c}\NormalTok{(}\DecValTok{1}\NormalTok{,}\DecValTok{2}\NormalTok{))}
\KeywordTok{plot}\NormalTok{(px, }\DataTypeTok{main=}\StringTok{'Maxent, raw values'}\NormalTok{)}
\KeywordTok{plot}\NormalTok{(wrld_simpl, }\DataTypeTok{add=}\OtherTok{TRUE}\NormalTok{, }\DataTypeTok{border=}\StringTok{'dark grey'}\NormalTok{)}
\NormalTok{tr <-}\StringTok{ }\KeywordTok{threshold}\NormalTok{(e, }\StringTok{'spec_sens'}\NormalTok{)}
\KeywordTok{plot}\NormalTok{(px >}\StringTok{ }\NormalTok{tr, }\DataTypeTok{main=}\StringTok{'presence/absence'}\NormalTok{)}
\KeywordTok{plot}\NormalTok{(wrld_simpl, }\DataTypeTok{add=}\OtherTok{TRUE}\NormalTok{, }\DataTypeTok{border=}\StringTok{'dark grey'}\NormalTok{)}
\KeywordTok{points}\NormalTok{(pres_train, }\DataTypeTok{pch=}\StringTok{'+'}\NormalTok{)}
\end{Highlighting}
\end{Shaded}

\includegraphics{index_files/figure-latex/unnamed-chunk-37-3.pdf}

Otro método que durante los últimos años se ha desarrollado es el método
Random Forest (Breiman, 2001b), este es una extensión de los árboles de
clasificación y regresión (CART; Breiman et al., 1984). En R se
implementa en la función `randomForest' en un paquete con el mismo
nombre. La función randomForest puede tomar una fórmula o, en dos
argumentos separados, un marco de datos con las variables predictoras y
un vector con la respuesta. Si la variable de respuesta es un factor
(categórico), randomForest hará una clasificación, de lo contrario hará
una regresión. Mientras que con el modelo de distribución de especies a
menudo estamos interesados en la clasificación (la especie está presente
o no), el uso de la regresión proporciona mejores resultados. rf1 hace
regresión, rf2 y rf3 hacen clasificación (son exactamente los mismos
modelos). Consulte la función tuneRF para optimizar el procedimiento de
ajuste del modelo.

\begin{Shaded}
\begin{Highlighting}[]
\KeywordTok{library}\NormalTok{(randomForest)}
\end{Highlighting}
\end{Shaded}

\begin{verbatim}
## Warning: package 'randomForest' was built under R version 3.5.3
\end{verbatim}

\begin{Shaded}
\begin{Highlighting}[]
\CommentTok{#Ajustamos el modelo}
\NormalTok{model <-}\StringTok{ }\NormalTok{pa ~}\StringTok{ }\NormalTok{bio16 +}\StringTok{ }\NormalTok{bio5 +}\StringTok{ }\NormalTok{bio6 +}\StringTok{ }\NormalTok{bio17 +}\StringTok{ }\NormalTok{bio12}
\NormalTok{rf1 <-}\StringTok{ }\KeywordTok{randomForest}\NormalTok{(model, }\DataTypeTok{data=}\NormalTok{envtrain)}
\end{Highlighting}
\end{Shaded}

\begin{verbatim}
## Warning in randomForest.default(m, y, ...): The response has five or fewer
## unique values. Are you sure you want to do regression?
\end{verbatim}

\begin{Shaded}
\begin{Highlighting}[]
\NormalTok{model <-}\StringTok{ }\KeywordTok{factor}\NormalTok{(pa) ~}\StringTok{ }\NormalTok{bio16 +}\StringTok{ }\NormalTok{bio5 +}\StringTok{ }\NormalTok{bio6 +}\StringTok{ }\NormalTok{bio17 +}\StringTok{ }\NormalTok{bio12}
\NormalTok{rf2 <-}\StringTok{ }\KeywordTok{randomForest}\NormalTok{(model, }\DataTypeTok{data=}\NormalTok{envtrain)}
\NormalTok{rf3 <-}\StringTok{ }\KeywordTok{randomForest}\NormalTok{(envtrain[,}\DecValTok{1}\NormalTok{:}\DecValTok{8}\NormalTok{], }\KeywordTok{factor}\NormalTok{(pb_train))}

\CommentTok{#Evaluamos el modelo usamos unicamente el primero}
\NormalTok{erf <-}\StringTok{ }\KeywordTok{evaluate}\NormalTok{(testpres, testbackg, rf1)}
\NormalTok{erf}
\end{Highlighting}
\end{Shaded}

\begin{verbatim}
## class          : ModelEvaluation 
## n presences    : 34 
## n absences     : 200 
## AUC            : 0.7736765 
## cor            : 0.3843336 
## max TPR+TNR at : 0.1684238
\end{verbatim}

\begin{Shaded}
\begin{Highlighting}[]
\CommentTok{#predecimos}
\NormalTok{pr <-}\StringTok{ }\KeywordTok{predict}\NormalTok{(predictors, rf1, }\DataTypeTok{ext=}\NormalTok{ext)}

\CommentTok{#graficamos}
\KeywordTok{par}\NormalTok{(}\DataTypeTok{mfrow=}\KeywordTok{c}\NormalTok{(}\DecValTok{1}\NormalTok{,}\DecValTok{2}\NormalTok{))}
\KeywordTok{plot}\NormalTok{(pr, }\DataTypeTok{main=}\StringTok{'Random Forest, regression'}\NormalTok{)}
\KeywordTok{plot}\NormalTok{(wrld_simpl, }\DataTypeTok{add=}\OtherTok{TRUE}\NormalTok{, }\DataTypeTok{border=}\StringTok{'dark grey'}\NormalTok{)}
\NormalTok{tr <-}\StringTok{ }\KeywordTok{threshold}\NormalTok{(erf, }\StringTok{'spec_sens'}\NormalTok{)}
\KeywordTok{plot}\NormalTok{(pr >}\StringTok{ }\NormalTok{tr, }\DataTypeTok{main=}\StringTok{'presence/absence'}\NormalTok{)}
\KeywordTok{plot}\NormalTok{(wrld_simpl, }\DataTypeTok{add=}\OtherTok{TRUE}\NormalTok{, }\DataTypeTok{border=}\StringTok{'dark grey'}\NormalTok{)}
\KeywordTok{points}\NormalTok{(pres_train, }\DataTypeTok{pch=}\StringTok{'+'}\NormalTok{)}
\KeywordTok{points}\NormalTok{(backg_train, }\DataTypeTok{pch=}\StringTok{'-'}\NormalTok{, }\DataTypeTok{cex=}\FloatTok{0.25}\NormalTok{)}
\end{Highlighting}
\end{Shaded}

\includegraphics{index_files/figure-latex/unnamed-chunk-38-1.pdf}

Support Vector Machines (SVMs; Vapnik, 1998) aplica un método lineal
simple a los datos, pero en un espacio de características de alta
dimensión no relacionado linealmente con el espacio de entrada, pero en
la práctica, no implica ningún cálculo en esa alta dimensión del
espacio. Esta simplicidad combinada con el rendimiento de vanguardia en
muchos problemas de aprendizaje (clasificación, regresión y detección de
novedad) ha contribuido a la popularidad de la SVM (Karatzoglou et al.,
2006). Fueron utilizados por primera vez en modelos de distribución de
especies por Guo et al. (2005). Existen varias implementaciones de svm
en R. Las implementaciones más útiles en nuestro contexto son
probablemente la función ``ksvm'' en el paquete ``kernlab''.

\begin{Shaded}
\begin{Highlighting}[]
\KeywordTok{library}\NormalTok{(kernlab)}
\end{Highlighting}
\end{Shaded}

\begin{verbatim}
## Warning: package 'kernlab' was built under R version 3.5.2
\end{verbatim}

\begin{verbatim}
## 
## Attaching package: 'kernlab'
\end{verbatim}

\begin{verbatim}
## The following objects are masked from 'package:raster':
## 
##     buffer, rotated
\end{verbatim}

\begin{Shaded}
\begin{Highlighting}[]
\NormalTok{svm <-}\StringTok{ }\KeywordTok{ksvm}\NormalTok{(pa ~}\StringTok{ }\NormalTok{bio16 +}\StringTok{ }\NormalTok{bio5 +}\StringTok{ }\NormalTok{bio6 +}\StringTok{ }\NormalTok{bio17 +}\StringTok{ }\NormalTok{bio12, }\DataTypeTok{data=}\NormalTok{envtrain)}
\NormalTok{esv <-}\StringTok{ }\KeywordTok{evaluate}\NormalTok{(testpres, testbackg, svm)}
\NormalTok{esv}
\end{Highlighting}
\end{Shaded}

\begin{verbatim}
## class          : ModelEvaluation 
## n presences    : 34 
## n absences     : 200 
## AUC            : 0.7601471 
## cor            : 0.3232351 
## max TPR+TNR at : 0.03340697
\end{verbatim}

\begin{Shaded}
\begin{Highlighting}[]
\CommentTok{#Hacemos la predicción}
\NormalTok{ps <-}\StringTok{ }\KeywordTok{predict}\NormalTok{(predictors, svm, }\DataTypeTok{ext=}\NormalTok{ext)}

\CommentTok{#Graficamos}
\KeywordTok{par}\NormalTok{(}\DataTypeTok{mfrow=}\KeywordTok{c}\NormalTok{(}\DecValTok{1}\NormalTok{,}\DecValTok{2}\NormalTok{))}
\KeywordTok{plot}\NormalTok{(ps, }\DataTypeTok{main=}\StringTok{'Support Vector Machine'}\NormalTok{)}
\KeywordTok{plot}\NormalTok{(wrld_simpl, }\DataTypeTok{add=}\OtherTok{TRUE}\NormalTok{, }\DataTypeTok{border=}\StringTok{'dark grey'}\NormalTok{)}
\NormalTok{tr <-}\StringTok{ }\KeywordTok{threshold}\NormalTok{(esv, }\StringTok{'spec_sens'}\NormalTok{)}
\KeywordTok{plot}\NormalTok{(ps >}\StringTok{ }\NormalTok{tr, }\DataTypeTok{main=}\StringTok{'presence/absence'}\NormalTok{)}
\KeywordTok{plot}\NormalTok{(wrld_simpl, }\DataTypeTok{add=}\OtherTok{TRUE}\NormalTok{, }\DataTypeTok{border=}\StringTok{'dark grey'}\NormalTok{)}
\KeywordTok{points}\NormalTok{(pres_train, }\DataTypeTok{pch=}\StringTok{'+'}\NormalTok{)}
\KeywordTok{points}\NormalTok{(backg_train, }\DataTypeTok{pch=}\StringTok{'-'}\NormalTok{, }\DataTypeTok{cex=}\FloatTok{0.25}\NormalTok{)}
\end{Highlighting}
\end{Shaded}

\includegraphics{index_files/figure-latex/unnamed-chunk-39-1.pdf}

\subsection{Combinando predicciones del
modelo}\label{combinando-predicciones-del-modelo}

En lugar de confiar en un solo ``mejor'' modelo, algunos autores (por
ejemplo, Thuillier, 2003) han abogado por utilizar muchos modelos y
aplicar algún tipo de promedio de modelos. Vea el paquete biomod2 para
una implementación. Por supuesto, puede implementar estos enfoques usted
mismo. A continuación se muestra un ejemplo muy breve. Primero hacemos
un RasterStack de nuestras predicciones de modelos individuales:

\begin{Shaded}
\begin{Highlighting}[]
\CommentTok{#modelos con AUC mayor a 0.70}
\NormalTok{models <-}\StringTok{ }\KeywordTok{stack}\NormalTok{( pg1, pg2,px,pr,ps)}
\KeywordTok{names}\NormalTok{(models) <-}\StringTok{ }\KeywordTok{c}\NormalTok{(}\StringTok{"glmLog"}\NormalTok{,}\StringTok{"glmGau"}\NormalTok{, }\StringTok{"maxent"}\NormalTok{, }\StringTok{"rf"}\NormalTok{, }\StringTok{"svm"}\NormalTok{)}
\KeywordTok{plot}\NormalTok{(models)}
\end{Highlighting}
\end{Shaded}

\includegraphics{index_files/figure-latex/unnamed-chunk-40-1.pdf}

\begin{Shaded}
\begin{Highlighting}[]
\CommentTok{#Calculamos la media}
\NormalTok{m <-}\StringTok{ }\KeywordTok{mean}\NormalTok{(models)}
\KeywordTok{plot}\NormalTok{(m, }\DataTypeTok{main=}\StringTok{'average score'}\NormalTok{)}
\end{Highlighting}
\end{Shaded}

\includegraphics{index_files/figure-latex/unnamed-chunk-41-1.pdf}

\begin{Shaded}
\begin{Highlighting}[]
\CommentTok{#como una media pesada por el AUC}

\NormalTok{auc <-}\StringTok{ }\KeywordTok{sapply}\NormalTok{(}\KeywordTok{list}\NormalTok{(emx,ge1, ge2, erf, esv), function(x) x@auc)}
\NormalTok{w <-}\StringTok{ }\NormalTok{(auc}\FloatTok{-0.5}\NormalTok{)^}\DecValTok{2}
\NormalTok{m2 <-}\StringTok{ }\KeywordTok{weighted.mean}\NormalTok{( models, w)}
\KeywordTok{plot}\NormalTok{(m2, }\DataTypeTok{main=}\StringTok{'weighted mean of three models'}\NormalTok{)}
\end{Highlighting}
\end{Shaded}

\includegraphics{index_files/figure-latex/unnamed-chunk-42-1.pdf}


\end{document}
